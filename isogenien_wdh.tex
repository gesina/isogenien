\section{Wiederholung}
\subsection{Morphismen von Kurven}

\begin{Definition}[Definiertheit]
  Sei $C$ eine Kurve, $P\in C$ und $\algK[C]_P$ die Lokalisierung
  ihres affinen Koordinatenrings nach dem Maximalideal $M_P$.
  Wir bezeichnen mit
  \begin{align*}
    \ord_P\colon \algK[C]_P &\longto \N_0 \cup \{\infty\} \\
    f &\mapsto \max\left\{d\in\Z\;\middle|\; f\in M_P^d \right\}
  \end{align*}
  die diskrete Bewertung auf $\algK[C]_P$ bzw. durch kanonische
  Erweiterung
  \begin{align*}
    \ord_P\colon \algK(C) &\longto \Z \cup \{\infty\} \\
    \frac{f}{g} &\longmapsto \ord_P(f) - \ord_P(g)
  \end{align*}
  die diskrete Bewertung auf dem Quotientenkörper $\algK(C)$.

  Anschaulich gibt $\ord_P(f)$ für $f\in\algK(C)$ an, ob $f$
  eine Nullstelle ($\ord_P(f)>0$) oder eine Polstelle ($\ord_P(f)<0$)
  in $P$ hat und mit welcher Vielfachheit ($=|\ord_P(f)|$).
  Insbesondere ist $f$ in $P$ auswertbar, wir bezeichnen es als
  definiert, wenn $\ord_P(f)\geq 0$.

  Für eine allgemeine projektive Varietät $V$ heißt eine Funktion
  $f\in\algK(V)$ definiert in $P\in V$, falls $f\in\algK[V]_P$.
\end{Definition}

\begin{Definition}[Morphismus projektiver Varietäten]
  Für projektive Varietäten $V_1\subset\P^m$, $V_2\subset\P^n$ heißt
  $\phi\colon V_1\to V_2$ eine rationale Abbildung,
  falls $\phi$ in der Form
  \begin{align*}
    \phi = [f_0,\dotsc, f_n]\colon V_1 &\longto V_2\\
    P &\longmapsto \left[f_0(P), \dotsc, f_n(P)\right] 
        \qquad \text{mit } f_i\in\algK(V_1)
  \end{align*}
  geschrieben werden kann, 
  wobei für jeden Punkt $P\in V_1$, 
  in dem alle $f_i$ definiert sind, % f_i in P auswertbar
  $(f_0(P),\dotsc,f_n(P))\neq0$. % Wohldefiniertheit
  Solch ein $\phi$ heißt definiert über $\K$,
  falls $f_i\in\K(V_1)$. 
  % Beachte hierbei die Invarianz
  % bzgl. Multiplikation mit Einheiten $\lambda$, denn
  % $[f_0,\dotsc,f_n]=[\lambda f_0,\dotsc,\lambda f_n]\in\P^n$.

  Eine rationale Abbildung $\phi$ heißt regulär in $P\in V_1$,
  falls es ein $g\in\algK(V)$ gibt, sodass
  alle $gf_i$ in $P$ definiert sind 
  und $(gf_0(P),\dotsc,gf_n(P))\neq0$.
  Beachte hierbei, 
  dass für ein in $P$ definiertes $g\in\algK(V)$ mit $g(P)\neq0$ gilt
  $[f_0,\dotsc,f_n]=[gf_0,\dotsc, gf_n]\in\P^n$
  (denn $[x_0,\dotsc,x_n]=[\lambda x_0,\dotsc,\lambda x_n]\in\P^n$ für
  $\lambda\in\algK$).
  
  Ein Morphismus von projektiven Varietäten ist eine rationale
  Abbildung $\phi\colon V_1\to V_2$, die in allen $P\in V_1$ regulär
  ist.
\end{Definition}

\begin{Lemma}\label{ratabbglattekurven}
  Sei $C\subset\P^2$ eine Kurve und $V\subset\P^n$ eine projektive
  Varietät. 
  Eine rationale Abbildung $\phi=[f_0,\dotsc, f_n]\colon C\to V$ ist
  regulär in allen regulären Punkten von $C$.
  Insbesondere sind rationale Abbildungen von glatten Kurven
  Morphismen.
  \begin{proof}[Beweisskizze]
    Für einen regulären Punkt $P\in C$ und einen Erzeuger
    $t\in\algK(C)$ von $M_P$ gilt mit 
    $r\coloneqq\min_{0\leq i\leq n}\{\ord_Pf_i\}$,
    dass $[t^{-r}f_0,\dotsc,t^{-r}f_n]$ regulär in $P$ ist.
    (Verwende die Tatsache, dass $g\in C$ genau dann regulär in $P$
    ist, wenn $\ord_P(g)\geq0$.)
  \end{proof}
\end{Lemma}

\begin{Lemma}\label{morphismensurj}
  Ein Morphismus von Kurven ist entweder konstant oder surjektiv.
  \begin{proof}
    \cite[siehe][Theorem II.2.3]{silverman}
  \end{proof}
\end{Lemma}

% ---

\subsection{Kategorie der glatten Kurven und Kategorienäquivalenz}
\begin{Bemerkung}[Kategorie der glatten Kurven]
  Wir erhalten die Kategorie der glatten Kurven
  mit nicht-konstanten, rationalen Abbildungen als Morphismen 
  (nach Lemma \ref{ratabbglattekurven} und \ref{morphismensurj} also
  surjektive Morphismen von Kurven)
\end{Bemerkung}

\begin{LemmaDefinition}\label{funktionenkoerper}
  Eine rationale Abbildung $\phi\colon C_1\to C_2$ von auf $\K$
  definierten Kurven induziert eine Körpererweiterung über $\K$ auf
  den Funktionenkörpern
  \begin{align*}
    \phi^* \colon \K(C_2) &\longto \K(C_1)\\
    f &\longmapsto f\circ \phi
  \end{align*}
  Diese Erweiterung ist endlich
  \cite[siehe][Theorem II.2.4 (a)]{silverman}.

  Man kann $\phi$ anhand der Eigenschaften der Körpererweiterung
  $\K(C_1)/\phi^*(\K(C_2))$ charakterisieren und erhält
  dementsprechend für $\phi$ nicht-konstant
  \begin{alignat*}{2}
    \deg(\phi) &\coloneqq \left[ \K(C_1) : \phi^*(\K(C_2)) \right]
    \qquad&&\text{(Erweiterungs-)Grad von }\phi\\
    \degs(\phi) &\coloneqq \left[ \K(C_1) : \phi^*(\K(C_2)) \right]_s
    \qquad&&\text{Separabilitätsgrad von }\phi\\
    \degi(\phi) &\coloneqq \left[ \K(C_1) : \phi^*(\K(C_2)) \right]_i
    \qquad&&\text{Inseparabilitätsgrad von }\phi
  \end{alignat*}
  bzw. die Bezeichnungen endlich/separabel/inseparabel/rein
  inseparabel. Zur Erinnerung: Es gelten die Zusammenhänge
  \begin{align*}
    \degs(\phi) 
    &\coloneqq \left[ \K(C_1) : \phi^*(\K(C_2)) \right]_s
    \coloneqq \#\Hom_K(\K(C_1), \overline{\K(C_2)})\\
    \deg(\phi) &= \degs(\phi)\cdot\degi(\phi)
  \end{align*}
  Wir setzen $\deg\phi\coloneqq 0$ für konstantes $\phi$.
\end{LemmaDefinition}

\begin{Lemma}\label{erweiterungzuratabb}
  Sei $\iota\colon \K(C_2)\to\K(C_1)$ eine Körpererweiterung von
  Funktionenkörpern zu über $K$ definierten, glatten Kurven $C_1$ und
  $C_2$, die $K$ fixiert. Dann gibt es eine 
  eindeutige, über $K$ definierte, nicht-konstante, rationale
  Abbildung $\lambda\colon C_1\to C_2$ mit $\lambda^* = \iota$.
  Ist $C_1$ glatt, ist $\lambda$ nach \autoref{ratabbglattekurven}
  ein Morphimus.
  \begin{proof}
    \begin{description}
      \item[Existenz] Sei \OE{} $C_2\in\P^n$ nicht in der Hyperebene
        $X_0=0$ enthalten und
        $g_i\coloneqq\overline{\left(\frac{X_i}{X_0}\right)}\in\K(C_2)$.
        Dann erfüllt die Abbildung
        \begin{gather*}
          \lambda\coloneqq \left[1, \iota(g_1),\dotsc,\iota(g_n)\right]
        \end{gather*}
        die gewünschte Eigenschaft, denn auf den Erzeugern $g_i$ gilt
        \begin{gather*}
          \iota(g_i) 
          = \frac{\iota(g_i)}{1} 
          = \left(\frac{X_i}{X_0}\right)\circ\lambda
          = g_i\circ\lambda
          = \lambda^*(g_i)
        \end{gather*}
        Sie ist nicht konstant, da die
        $g_i$ nicht alle konstant sind und $\iota$ injektiv ist, also
        die $\iota(g_i)$ ebenfalls nicht alle konstant sind. Und sie
        ist offensichtlich rational sowie über $\K$ definiert.
      \item[Eindeutigkeit]
        Sei $\tlambda\coloneqq[f_0,\dotsc,f_n]$ eine weitere
        Abbildungen mit den gesuchten Eigenschaften.
        Dann gilt
        \begin{gather*}
          \frac{f_i}{f_0}
          %= \left(\frac{X_i}{X_0}\right)\circ\tlambda 
          = g_i\circ\tlambda
          = \tlambda^*(g_i) 
          \overset{\text{Annahme}}{=}
          \iota(g_i)
        \end{gather*}
        also 
        \begin{gather*}
          \tlambda
          \coloneqq \left[ f_0,f_1,\dotsc,f_n \right]
          = \left[ 1, \frac{f_1}{f_0}, \dotsc, \frac{f_n}{f_0}\right]
          = \left[ 1, \iota(g_1), \dotsc, \iota(g_n) \right]
          \eqqcolon \lambda
        \end{gather*}
    \end{description}
  \end{proof}
\end{Lemma}

\begin{Satz}[Kategorienäquivalenz]\label{kategorienaequivalenz}
  Die Aussagen von oben lassen sich zu einer Kategorienantiäquivalenz
  zwischen der Kategorie der glatten Kurven und der Kategorie der
  Körpererweiterungen $\K'$ über $\K$ mit Transzendenzgrad 1 und
  $\K'\cap\algK=\K$ verallgemeinern:
  \begin{align*}
    C/K &\Longrightarrow \K(C)\\
    \phi\colon C_1\to C_2 &\Longrightarrow \phi^*\colon\K(C_2)\to\K(C_1)
  \end{align*}
  \cite[siehe][Corollary I.6.11]{hartshorne} und
  \cite[siehe][Remark II.2.5]{silverman}.
  %% Körpererweiterungszeugs: siehe S.24–25
\end{Satz}

%---

\subsection{Verzweigungsindex}
\begin{Definition}[Verzweigungsindex]\label{verzweigungsindex}
  Sei $\phi\colon C_1\to C_2$ eine nicht-konstante Abbildung zwischen
  glatten Kurven.
  Dann ist der Verzweigungsindex von $\phi$ in einem Punkt
  $P\in C_1$ definiert als
  \begin{gather*}
    e_\phi(P) = \ord_P \left( \phi^*(t_{\phi(P)}) \right) \geq 1
  \end{gather*}
  wobei $t_{\phi(P)}$ ein Erzeuger von $M_{\phi(P)}$ ist (d.\,h. auch
  eindeutig bis auf Multiplikation mit Einheiten, da $P$ regulär ist).
\end{Definition}

\begin{Satz}\label{sepgrad}
  Sei $\phi\colon C_1\to C_2$ wieder eine nicht-konstante Abbildung
  zwischen glatten Kurven. 
  \begin{enumerate}[label=\roman*)]
    \item Es gilt für alle $Q\in C_2$
      \begin{gather*}
        \sum_{\mathclap{P\in\phi^{-1}(Q)}} e_\phi(P) = \deg\phi
      \end{gather*}
      Insbesondere gilt, da $\deg\phi$ endlich ist
      (s. \ref{funktionenkoerper}) und $e_\phi(P)\geq 1$,
      dass das Urbild $\phi^{-1}(Q)$ jedes Punkts $Q\in C_2$
      endlich viele Punkte in $C_1$ enthält.
    \item Für alle außer endlich viele Punkte $Q\in C_2$ gilt
      \begin{gather*}
        \#\phi^{-1}(Q) = \degs(\phi)
      \end{gather*}
    \item Der Verzweigungsindex ist multiplikativ
      in folgendem Sinne
      \begin{gather*}
        e_{\psi\circ\phi}(P) = e_\phi(P) \cdot e_\psi(\phi(P))
      \end{gather*}
    \end{enumerate}
    \begin{proof}
      \cite[siehe][Proposition II.2.6]{silverman}
      
      % zu \emph{ii)}: Nach Definition des Separabilitätsgrades gilt
      % (wir identifizieren $\phi^*(\algK(C_2))$ mit $\algK(C_2)$)
      % \begin{gather*}
      %   \degs(\phi) 
      %   \coloneqq \degs\left( \algK(C_1)/\algK(C_2) \right)
      %     \coloneqq
      %   \#\Hom_{\algK(C_2)}\left( \algK(C_1), \overline{\algK(C_2)} \right)
      % \end{gather*}
      % Mit unserer Kategorienantiäquivalenz aus
      % \autoref{kategorienaequivalenz} erhalten wir zur
      % $\algK$-Erweiterung $\overline{\algK(C_2)}$ eine glatte Kurve
      % $C_0$ mit einem Morphismus $\lambda\colon C_0\to C_2$, für das
      % $\lambda^*$ die Erweiterung $\algK(C_2)\to\overline{\algK(C_2)}$
      % ist (s. \autoref{erweiterungzuratabb}).
      % $\Hom_{\algK(C_2)}(\algK(C_1), \overline{\algK(C_2)})$
      % wird zur Menge der Homomorphismen $\psi$, die über $\phi$ wie
      % folgt faktorisieren:
      % \begin{center}
      %   \begin{tikzcd}
      %     C_0
      %     \arrow[r, "\psi"{below}]
      %     \arrow[rr, "\lambda"{below}, bend left=30]
      %     & C_1
      %     \arrow[r, "\phi"{below}]
      %     & C_2
      %     \\
      %     \algK(C_0)=\overline{\algK(C_2)}
      %     \arrow[from=r, "\psi^*"{above}]
      %     \arrow[from=rr, "\lambda^*"{above}, bend left=30]
      %     & \algK(C_1)
      %     \arrow[from=r, "\phi^*"{above}]
      %     & \algK(C_2)
      %   \end{tikzcd}
      % \end{center}
      % Sei $P\in C_2$. Dann 

    \end{proof}


\end{Satz}

% ---

\subsection{Divisorengruppe}
\begin{Definition}[Divisorengruppe]
  Für eine Kurve $C$ wird die von den Punkten von $C$ erzeugte, freie
  abelsche Gruppe mit $\Div(C)$ bezeichnet und ihre
  Elemente heißen Divisoren. Ein Punkt $P$ erhält als Divisor die
  Schreibweise $(P)$. 
  Der Grad eines Divisors $D=\sum_{P\in C}n_P\cdot(P)$ ist definiert als
  \begin{gather*}
    \deg D \coloneqq \sum_{P\in C} n_P
  \end{gather*}
  
  Ist $C$ glatt, definieren wir für eine Funktion
  $f\in\algK(C)^*$ den zugehörigen Divisor $\div(f)$
  als
  \begin{gather*}
    \div(f) \coloneqq \sum_{P\in C}\ord_P(f)\cdot(P)
  \end{gather*}
  $\div\colon\algK(C)^*\to\Div(C)$ ist ein Gruppenhomomorphismus,
  d.\,h. $\div(f\cdot g)=\div(f)+\div(g)$,
  mit Kern $\algK^*$.
  Ein Element im Bild von $\div$ wird prinzipal genannt.
  
  Die Picard-Gruppe $\Pic(C)$ von $C$ ist der Quotient
  $\Div(C)/\div(\algK(C)^*)$ der Divisorengruppe über der Untergruppe
  der prinzipalen Elemente. Wir schreiben 
  $(P)\sim(Q):\Leftrightarrow (P)-(Q)=\div(f)$ für ein $f\in\algK(C)^*$.

  Wir definieren die Grad-0-Untergruppe $\Div^0(C)\subset\Div(C)$ als
  \begin{gather*}
    \Div^0(C) \coloneqq \left\{ D\in\Div(C) \;\middle|\; \deg D=0 \right\}
  \end{gather*}
  und entsprechend $\Pic^0(C)$ über die exakte Sequenz
  \begin{center}
    \begin{tikzcd}
      1 \arrow[r] &\algK^* \arrow[r] &\algK(C)^* \arrow[r, "\div"]
      &\Div^0(C) \arrow[r] &\Pic^0(C) \arrow[r] &0
    \end{tikzcd}
  \end{center}
\end{Definition}

Aus der Algebraischen Geometrie ist bekannt, dass die Punkte einer
Elliptischen Kurve $E$ mit einer geometrischen Addition eine Gruppe
bilden.

\begin{Satz}\label{gruppenaequivalenz}
  Eine Elliptische Kurve sei als Gruppe ihrer Punkte aufgefasst
  mit der geometrischen Addition.
  \begin{enumerate}[label=\roman*)]
  \item 
    Für eine Elliptische Kurve $E$ mit neutralem Element $O\in E$ gibt
    es einen Gruppenisomorphismus
    \begin{align*}
      \sigma\colon \Pic^0(E) &\longto E\\
      \overline{(D)} &\mapsto P
                       \quad\text{mit}\quad D\sim (P)-(O)\\
      \overline{(P)-(O)} &\longmapsfrom P : \kappa
    \end{align*}

  \item
    Jeder nicht-konstante Morphismus $\phi\colon E_1\to E_2$, 
    auf elliptischen Kurven $E_1$, $E_2$ induziert Gruppenhomomorphismen
    \begin{alignat*}{3}
      &\phi_*\colon& \Div(E_1) 
      &\longto \Div(E_2),
      \qquad
      &(P) &\longmapsto (\phi(P))\\
      &\phi_*\colon& \Div^0(E_1) 
      &\longto \Div^0(E_2),
      &(P) &\longmapsto (\phi(P))\\
      &\phi_*\colon& \Pic^0(E_1) 
      &\longto \Pic^0(E_2),
      &\overline{(P)} &\longmapsto \overline{(\phi(P))}\\
    \end{alignat*}
  \end{enumerate}
  \begin{proof}
    Zu \emph{i)} \cite[siehe][Proposition III.3.4]{silverman} und
    zu \emph{ii)} \cite[siehe][Remark II.3.7]{silverman}.
    \end{proof}
  \end{Satz}
  Wir werden die beiden Gruppen im Folgenden immer miteinander
  identifizieren.


  \begin{Satz}\label{additionmorphismus}
    Substraktion und Addition (bzw. Translation $\tau_Q$ um $Q\in E$)auf
    Elliptischen Kurven sind Morphismen von Kurven. Die Abbildungen sind:
    \begin{align*}
      \tau_Q\coloneqq\bullet+Q\colon E\times E&\longto E  
      &&\text{und}  
      &-\bullet\colon E&\longto E \\
      P&\longmapsto P+Q             
      &&&P&\longmapsto -P
    \end{align*}
    \begin{proof}
      \cite[siehe][Theorem III.3.6]{silverman}
    \end{proof}
  \end{Satz}


  Eine weiterer wichtiger Zusammenhang ist die Hurwitz'sche Formel für
  das Geschlecht von Kurven.

  \begin{Satz}[Hurwitz]\label{hurwitz}
    Für einen separablen Morphismus $\phi\colon C_1\to C_2$ von
    glatten Kurven 
    je mit Geschlecht $g_i$ gilt
    \begin{gather*}
      2g_1-2
      \geq
      \deg(\phi)\cdot(2g_2-2) + \sum_{\mathclap{P\in C_1}}(e_P(\phi)-1)
    \end{gather*}
    über einem Körper $\K$ mit 
    $\Char(\K)\nmid e_P(\phi)$ für alle $P\in C_1$ 
    gilt Gleichheit.
    \begin{proof}
      \cite[siehe][Theorem II.5.9]{silverman}
    \end{proof}
  \end{Satz}


  %%% Local Variables:
  %%% mode: latex
  %%% TeX-master: "isogenien_ausarbeitung"
  %%% End:
