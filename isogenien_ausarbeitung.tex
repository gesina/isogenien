\documentclass[english, german]{scrartcl}
\usepackage[utf8]{inputenc}
\usepackage[T1]{fontenc}
\usepackage{babel}
\usepackage{csquotes}

\usepackage{lmodern}

\usepackage[backend=biber]{biblatex}
\bibliography{isogenien_literatur.bib}

\usepackage{mathtools}
\usepackage{amssymb}
\usepackage{amsthm}
\usepackage{dsfont}

\usepackage{enumitem}

% DEFINITIONS
\newtheorem{Satz}{Satz}[section]
\newtheorem{Theorem}[Satz]{Theorem}
\newtheorem{Proposition}[Satz]{Proposition}
\newtheorem{Lemma}[Satz]{Lemma}
\theoremstyle{definition}
\newtheorem{Definition}[Satz]{Definition}
\theoremstyle{remark}
\newtheorem{Bemerkung}[Satz]{Bemerkung}

\newcommand*{\K}{\ensuremath{K}}
\newcommand*{\algK}{\ensuremath{\overline K}}
\renewcommand*{\P}{\ensuremath{\mathds{P}}}
\newcommand*{\longto}{\longrightarrow}
\newcommand*{\Xn}{\underline{X}}
\DeclareMathOperator{\ord}{ord}

% TITEL
\titlehead{Seminar Elliptische Kurven \\
  von Prof. Kerz,
  SS2016, Universität Regensburg}
\subject{Vortragsausarbeitung}
\title{Isogenien}
\author{Gesina Schwalbe}

\begin{document}
\maketitle
\tableofcontents

\section{Vorwort}

\subsection{Nomenklatur}
\begin{description}[labelwidth=1.5cm, font=\normalfont, itemsep=0pt]
  \item[$\K$] Körper
  \item[$\algK$] algebraischer Abschluss des Körpers $K$
  \item[${\K[\Xn]}$] $\K[X_0,\dotsc,X_n]$
  \item[$\P^n$] projektiver Raum der Dimension $n$
  \item[$V(I)$] projektive Varietät erzeugt vom homogenen Ideal $I$
  \item[$I(V)$] homogenes Ideal $I$, das die projektive Varietät $V$ erzeugt
  \item[$V/K$] projektive Varietät definiert über $\K$ (d.\,h. $I(V)$
    ist erzeugt von Polynomen in $\K(X)$
  \item[$P\in V$] Punkt der projektiven Varietät $V$ oder 
    das Maximalideal 
    \begin{gather*}
      M_P\coloneqq\left\{f\in\algK[V]\;\middle\vert\;f(P)=0\right\}
      \subset \algK[V]
    \end{gather*}
  \item[{$\algK[V]$}] affiner Koordinatenring der projektiven Varietät
    $V$ ($\algK[V]\coloneqq \algK[\Xn]/I(V)$)
  \item[{$\algK(V)$}] Quotientenkörper von $\algK[V]$
  \item[$C$] projektive Kurve
  \item[$\ord_P(f)$] Ordnung von $f\in\algK(C)$ bzgl. der diskreten
    Bewertung auf $\algK[C]_P$
\end{description}

\section{Einführung}
\begin{Definition}[Morphismus projektiver Varietäten]
  Für projektive Varietäten $V_1\subset\P^m$, $V_2\subset\P^n$ heißt
  $\phi\colon V_1\to V_2$ eine \emph{rationale Abbildung},
  falls $\phi$ in der Form
  \begin{align*}
    \phi = [f_0,\dotsc, f_n]\colon V_1 &\longto V_2\\
    P &\longmapsto \left[f_0(P), \dotsc, f_n(P)\right] 
        \qquad \text{mit } f_i\in\algK(V_1)
  \end{align*}
  geschrieben werden kann, 
  wobei für jeden Punkt $P\in V_1$, 
  in dem alle $f_i$ definiert sind, % f_i in P auswertbar
  $(f_0(P),\dotsc,f_n(P))\neq0$. % Wohldefiniertheit
  
  Solch ein $\phi$ heißt \emph{definiert über $\K$},%
  falls $f_i\in\K(V_1)$. Beachte hierbei die Invarianz
  bzgl. Multiplikation mit Einheiten $\lambda$, denn
  $[f_0,\dotsc,f_n]=[\lambda f_0,\dotsc,\lambda f_n]\in\P^n$.

  Eine rationale Abbildung $\phi$ heißt \emph{regulär in $P\in V_1$},
  falls es ein $g\in\algK(V)$ gibt, sodass
  alle $gf_i$ in $P$ definiert sind 
  und $(gf_0(P),\dotsc,gf_n(P))\neq0$.
  Beachte hierbei, 
  dass für ein in $P$ definiertes $g\in\algK(V)$ mit $g(P)\neq0$ gilt
  $[f_0,\dotsc,f_n]=[gf_0,\dotsc, gf_n]\in\P^n$.
  
  Ein \emph{Morphismus von projektiven Varietäten} ist eine rationale
  Abbildung $\phi\colon V_1\to V_2$, die in allen $P\in V_1$ regulär
  ist.
\end{Definition}

\begin{Lemma}\label{ratabbglattekurven}
  Sei $C\subset\P^2$ eine Kurve und $V\subset\P^n$ eine projektive
  Varietät. 
  Eine rationale Abbildung $\phi=[f_0,\dotsc, f_n]\colon C\to V$ ist
  regulär in allen regulären Punkten von $C$.
  Insbesondere sind rationale Abbildungen von glatten Kurven
  Morphismen.
  \begin{proof}[Beweisskizze]
    Für einen regulären Punkt $P\in C$ und einen Erzeuger
    $t\in\algK(C)$ von $M_P$ gilt mit 
    $r\coloneqq\min_{0\leq i\leq n}\{\ord_Pf_i\}$,
    dass $[t^{-r}f_0,\dotsc,t^{-r}f_n]$ regulär in $P$ ist.
    (Verwende die Tatsache, dass $g\in C$ regulär in $P$ ist
    genau dann, wenn $\ord_P(g)\geq0$.)
  \end{proof}
\end{Lemma}

\begin{Lemma}\label{morphismensurj}
  Ein Morphismus von Kurven ist entweder konstant oder surjektiv.
  \begin{proof}
    ??
  \end{proof}
\end{Lemma}

\begin{Bemerkung}[Kategorie der glatten Kurven]
  Wir erhalten die \emph{Kategorie der glatten Kurven}:
  \begin{description}
  \item[Objekte] glatte Kurven
  \item[Morphismen] surjektive Morphismen (nach Lemma
    \ref{ratabbglattekurven} und \ref{morphismensurj} also
    nicht-konstante, rationale Abbildungen)
  \end{description}
\end{Bemerkung}

\begin{Bemerkung}[Kategorienäquivalenz]
  Eine rationale Abbildung $\phi\colon C_1\to C_2$ auf Kurven, die auf
  $\K$ definiert sind,
  induziert eine injektive Abbildungen auf den Funktionenkörpern
  \begin{align*}
    \phi^* \colon \K(C_2) &\longto \K(C_1)\\
    f &\longmapsto f\circ \phi
  \end{align*}
  Dies liefert eine Kategorienantiäquivalenz zwischen der Kategorie der
  glatten Kurven und der Kategorie der Körpererweiterungen $\K'$ über $\K$
  mit Transzendenzgrad 1 und $\K'\cap\algK=\K$:
  \begin{align*}
    C/K &\Longrightarrow \K(C)\\
    \phi\colon C_1\to C_2 &\Longrightarrow \phi^*\colon\K(C_2)\to\K(C_1)
  \end{align*}
%% Körpererweiterungszeugs: siehe S.24–25
\end{Bemerkung}


% ramification index, Prop.2.6a) (S. 28)
% Divisorengruppe/Punktegruppe, Translationsabbildung


\section{Definition, Basic Properties}
\section{Multiplication by m map}
\section{m-Torsions-Untergrupppe}
\section{Isogenien sind Homomorphismen}
\section{Galois-Theorie elliptischer Funktionenkörper}
\section{Eindeutige Faktorisierung}
\section{Modulorechnung von Elliptischen Kurven}

\nocite{*}
\printbibliography
\end{document}
