\documentclass[english, german, parskip=half]{scrartcl}
\usepackage[utf8]{inputenc}
\usepackage[T1]{fontenc}
\usepackage{babel}
\usepackage{csquotes}

\usepackage{lmodern}

\usepackage[backend=biber]{biblatex}
\bibliography{isogenien_literatur.bib}

\usepackage{mathtools}
\usepackage{amssymb}
\usepackage{amsthm}
\usepackage{dsfont}
\usepackage{stmaryrd}
\usepackage{tikz-cd}
\usetikzlibrary{babel}

\usepackage{enumitem}

\usepackage{hyperref}
\hypersetup{
  pdfauthor={Gesina Schwalbe},
  pdftitle={Isogenien Seminarausarbeitung}
}

% DEFINITIONS
\newtheorem{Satz}{Satz}[section]
\newtheorem{Theorem}[Satz]{Theorem}
\newtheorem{Proposition}[Satz]{Proposition}
\newtheorem{Lemma}[Satz]{Lemma}
\newtheorem{Korollar}[Satz]{Korollar}
\newtheorem{Beispiel}[Satz]{Beispiel}
\theoremstyle{definition}
\newtheorem{Definition}[Satz]{Definition}
\newtheorem{LemmaDefinition}[Satz]{Lemma/Definition}
\theoremstyle{remark}
\newtheorem{Bemerkung}[Satz]{Bemerkung}

\newcommand*{\N}{\mathds{N}}
\newcommand*{\Z}{\mathds{Z}}
\newcommand*{\K}{\ensuremath{K}} % Körper
\newcommand*{\algK}{\ensuremath{\overline K}} % algebraischer Abschluss
\renewcommand*{\P}{\ensuremath{\mathds{P}}} % projektiver Raum
\newcommand*{\longto}{\longrightarrow}
\newcommand*{\longfrom}{\longleftarrow}
\newcommand*{\Xn}{\underline{X}} % Variablenmenge
\newcommand*{\degs}{\operatorname{\deg}_s} % Separabilitätsgrad
\newcommand*{\degi}{\operatorname{\deg}_i} % Inseparabilitätsgrad
\DeclareMathOperator{\ord}{ord} % diskrete Bewertung
\DeclareMathOperator{\Div}{Div} % Divisorengruppe
\DeclareMathOperator{\Pic}{Pic} % Picard Gruppe
\renewcommand{\div}{\operatorname{div}}
\DeclareMathOperator{\Hom}{Hom}
\DeclareMathOperator{\End}{End}
\DeclareMathOperator{\Aut}{Aut}
\newcommand{\Id}{\mathrm{id}}
\DeclareMathOperator{\im}{im} % image
\newcommand{\tlambda}{\tilde\lambda} % alternative Abbildung \lambda

% TITEL
\titlehead{Seminar Elliptische Kurven \\
  von Prof. Kerz,
  SS2016, Universität Regensburg}
\subject{Vortragsausarbeitung}
\title{Isogenien}
\author{Gesina Schwalbe}

\begin{document}
\maketitle
\tableofcontents

\section{Vorwort}

\subsection{Nomenklatur}
\begin{description}[labelwidth=1.5cm, font=\normalfont, itemsep=0pt]
  \item[$\K$] Körper
  \item[$\algK$] algebraischer Abschluss des Körpers $K$
  \item[${\K[\Xn]}$] $\K[X_0,\dotsc,X_n]$
  \item[$\P^n$] projektiver Raum der Dimension $n$
  \item[$V(I)$] projektive Varietät erzeugt vom homogenen Ideal $I$
  \item[$I(V)$] homogenes Ideal $I$, das die projektive Varietät $V$ erzeugt
  \item[$V/K$] projektive Varietät definiert über $\K$ (d.\,h. $I(V)$
    ist erzeugt von Polynomen in $\K(X)$
  \item[$P\in V$] Punkt der projektiven Varietät $V$ oder 
    das Maximalideal 
    \begin{gather*}
      M_P\coloneqq\left\{f\in\algK[V]\;\middle\vert\;f(P)=0\right\}
      \subset \algK[V]
    \end{gather*}
  \item[{$\algK[V]$}] affiner Koordinatenring der projektiven Varietät
    $V$ ($\algK[V]\coloneqq \algK[\Xn]/I(V)$)
  \item[{$\algK(V)$}] Quotientenkörper von $\algK[V]$
  \item[$C$] projektive Kurve
  \item[$\ord_P(f)$] Ordnung von $f\in\algK(C)$ bzgl. der diskreten
    Bewertung auf $\algK[C]_P$
\end{description}

\subsection{Motivation}
Im Folgenden werden wir eine bestimmte

\section{Wiederholung}
\subsection{Morphismen von Kurven}
Im folgenden ist mit Kurve immer eine projektive Varietät der
Dimension 1 gemeint und mit Elliptische Kurve eine Kurve $E$ vom
Geschlecht 1 zusammen mit einem fixen Punkt $O\in E$.

\begin{Definition}[Definiertheit]
  Sei $C$ eine Kurve, $P\in C$ und $\algK[C]_P$ die Lokalisierung
  ihres affinen Koordinatenrings nach dem Maximalideal $M_P$.
  Wir bezeichnen mit
  \begin{align*}
    \ord_P\colon \algK[C]_P &\longto \N_0 \cup \{\infty\} \\
    f &\mapsto \max\left\{d\in\Z\;\middle|\; f\in M_P^d \right\}
  \end{align*}
  die diskrete Bewertung auf $\algK[C]$ bzw. durch kanonische
  Erweiterung
  \begin{align*}
    \ord_P\colon \algK(C) &\longto \Z \cup \{\infty\} \\
    \frac{f}{g} &\longmapsto \ord_P(f) - \ord_P(g)
  \end{align*}
  die diskrete Bewertung auf dem Quotientenkörper $\algK(C)$.

  Anschaulich gibt $\ord_P(f)$ für $f\in\algK(C)$ an, ob $f$
  eine Nullstelle ($\ord_P(f)>0$) oder eine Polstelle ($\ord_P(f)<0$)
  in $P$ hat und mit welcher Vielfachheit ($=|\ord_P(f)|$).
  Insbesondere ist $f$ in $P$ auswertbar, wir bezeichnen es als
  definiert, wenn $\ord_P(f)\geq 0$.

  Für eine allgemeine projektive Varietät $V$ heißt eine Funktion
  $f\in\algK(V)$ definiert in $P\in V$, falls $f\in\algK[V]_P$.
\end{Definition}

\begin{Definition}[Morphismus projektiver Varietäten]
  Für projektive Varietäten $V_1\subset\P^m$, $V_2\subset\P^n$ heißt
  $\phi\colon V_1\to V_2$ eine rationale Abbildung,
  falls $\phi$ in der Form
  \begin{align*}
    \phi = [f_0,\dotsc, f_n]\colon V_1 &\longto V_2\\
    P &\longmapsto \left[f_0(P), \dotsc, f_n(P)\right] 
        \qquad \text{mit } f_i\in\algK(V_1)
  \end{align*}
  geschrieben werden kann, 
  wobei für jeden Punkt $P\in V_1$, 
  in dem alle $f_i$ definiert sind, % f_i in P auswertbar
  $(f_0(P),\dotsc,f_n(P))\neq0$. % Wohldefiniertheit
  
  Solch ein $\phi$ heißt definiert über $\K$,%
  falls $f_i\in\K(V_1)$. Beachte hierbei die Invarianz
  bzgl. Multiplikation mit Einheiten $\lambda$, denn
  $[f_0,\dotsc,f_n]=[\lambda f_0,\dotsc,\lambda f_n]\in\P^n$.

  Eine rationale Abbildung $\phi$ heißt regulär in $P\in V_1$,
  falls es ein $g\in\algK(V)$ gibt, sodass
  alle $gf_i$ in $P$ definiert sind 
  und $(gf_0(P),\dotsc,gf_n(P))\neq0$.
  Beachte hierbei, 
  dass für ein in $P$ definiertes $g\in\algK(V)$ mit $g(P)\neq0$ gilt
  $[f_0,\dotsc,f_n]=[gf_0,\dotsc, gf_n]\in\P^n$.
  
  Ein Morphismus von projektiven Varietäten ist eine rationale
  Abbildung $\phi\colon V_1\to V_2$, die in allen $P\in V_1$ regulär
  ist.
\end{Definition}

\begin{Lemma}\label{ratabbglattekurven}
  Sei $C\subset\P^2$ eine Kurve und $V\subset\P^n$ eine projektive
  Varietät. 
  Eine rationale Abbildung $\phi=[f_0,\dotsc, f_n]\colon C\to V$ ist
  regulär in allen regulären Punkten von $C$.
  Insbesondere sind rationale Abbildungen von glatten Kurven
  Morphismen.
  \begin{proof}[Beweisskizze]
    Für einen regulären Punkt $P\in C$ und einen Erzeuger
    $t\in\algK(C)$ von $M_P$ gilt mit 
    $r\coloneqq\min_{0\leq i\leq n}\{\ord_Pf_i\}$,
    dass $[t^{-r}f_0,\dotsc,t^{-r}f_n]$ regulär in $P$ ist.
    (Verwende die Tatsache, dass $g\in C$ regulär in $P$ ist
    genau dann, wenn $\ord_P(g)\geq0$.)
  \end{proof}
\end{Lemma}

\begin{Lemma}\label{morphismensurj}
  Ein Morphismus von Kurven ist entweder konstant oder surjektiv.
  \begin{proof}
    \cite[siehe][Theorem 2.3]{silverman}
  \end{proof}
\end{Lemma}

%---

\subsection{Verzweigungsindex}
\begin{Bemerkung}[Kategorie der glatten Kurven]
  Wir erhalten die Kategorie der glatten Kurven:
  \begin{description}
  \item[Objekte] glatte Kurven
  \item[Morphismen] surjektive Morphismen (nach Lemma
    \ref{ratabbglattekurven} und \ref{morphismensurj} also
    nicht-konstante, rationale Abbildungen)
  \end{description}
\end{Bemerkung}

\begin{Bemerkung}[Kategorienäquivalenz]\label{kategorienaequivalenz}
  Eine rationale Abbildung $\phi\colon C_1\to C_2$ auf Kurven, die auf
  $\K$ definiert sind,
  induziert eine injektive Abbildungen auf den Funktionenkörpern
  \begin{align*}
    \phi^* \colon \K(C_2) &\longto \K(C_1)\\
    f &\longmapsto f\circ \phi
  \end{align*}

  Man kann $\phi$ anhand der Eigenschaften der Körpererweiterung
  $\K(C_1)/\phi^*(\K(C_2))$ charakterisieren und erhält
  dementsprechend für $\phi$ nicht konstant
  \begin{gather*}
    \deg\phi \coloneqq \left[ \K(C_1) : \phi^*(\K(C_2)) \right]
  \end{gather*}
  bzw. die Bezeichnungen endlich/separabel/inseparabel/rein
  inseparabel.
  Wir setzen $\deg\phi\coloneqq 0$ für konstantes $\phi$.
  Es gilt allgemein, dass $\deg\phi$ endlich ist.
  \cite[siehe][Theorem 2.4 (a)]{silverman}.

  Dies liefert eine Kategorienantiäquivalenz zwischen der Kategorie der
  glatten Kurven und der Kategorie der Körpererweiterungen $\K'$ über $\K$
  mit Transzendenzgrad 1 und $\K'\cap\algK=\K$:
  \begin{align*}
    C/K &\Longrightarrow \K(C)\\
    \phi\colon C_1\to C_2 &\Longrightarrow \phi^*\colon\K(C_2)\to\K(C_1)
  \end{align*}
  
%% Körpererweiterungszeugs: siehe S.24–25
\end{Bemerkung}

\begin{Definition}[Verzweigungsindex]
  Sei $\phi\colon C_1\to C_2$ eine nicht-konstante Abbildung zwischen
  glatten Kurven.
  Dann ist der Verzweigungsindex von $\phi$ in einem Punkt
  $P\in C_1$ definiert als
  \begin{gather*}
    e_\phi(P) = \ord_P \left( \phi^*(t_{\phi(P)}) \right) \geq 1
  \end{gather*}
  wobei $t_{\phi(P)}$ ein Erzeuger von $M_{\phi(P)}$ ist (d.\,h. auch
  eindeutig bis auf Multiplikation mit Einheiten, da $P$ regulär ist).
\end{Definition}

\begin{Satz}
  Sei $\phi\colon C_1\to C_2$ wieder eine nicht-konstante Abbildung
  zwischen glatten Kurven. Dann gilt für alle $Q\in C_2$
  \begin{gather*}
    \sum_{\mathclap{P\in\phi^{-1}(Q)}} e_\phi(P) = \deg\phi
  \end{gather*}
  Insbesondere gilt, da $\deg\phi$ endlich ist
  (s. \ref{kategorienaequivalenz}) und $e_\phi(P)\geq 1$,
  dass das Urbild $\phi^{-1}(Q)$ jedes Punkts $Q\in C_2$
  endlich viele Punkte in $C_1$ enthält.
  \begin{proof}
    \cite[siehe][Proposition 2.6 (a)]{silverman}
  \end{proof}
\end{Satz}

%---

\subsection{Divisorengruppe}
\begin{Definition}[Divisorengruppe]
  Für eine Kurve $C$ wird die von den Punkten von $C$ erzeugte, freie
  abelsche Gruppe als die Gruppe $\Div(C)$ bezeichnet und ihre
  Elemente heißte Divisoren. Ein Punkt $P$ erhält als Divisor die
  Schreibweise $(P)$. 
  Der Grad eines Divisors $D=\sum_{P\in C}n_P\cdot(P)$ ist definiert als
  \begin{gather*}
    \deg D \coloneqq \sum_{P\in C} n_P
  \end{gather*}
  
  Ist $C$ glatt, definieren wir für eine Funktion
  $f\in\algK(C)^*$ den zugehörigen Divisor $\div(f)$
  als
  \begin{gather*}
    \div(f) \coloneqq \sum_{P\in C}\ord_P(f)\cdot(P)
  \end{gather*}
  $\div\colon\algK(C)^*\to\Div(C)$ ist ein Gruppenhomomorphismus,
  d.\,h. $\div(f\cdot g)=\div(f)+\div(g)$,
  mit Kern $\algK^*$.
  Eine Element im Bild von $\div$ wird prinzipal genannt.
  
  Die Picard-Gruppe $\Pic(C)$ von $C$ ist der Quotient
  $\Div(C)/\div(\algK(C)^*)$ der Divisorengruppe über der Untergruppe
  der prinzipalen Elemente. Wir definieren 
  $(P)\sim(Q)\Leftrightarrow (P)-(Q)=\div(f)$ für ein $f\in\algK(C)^*$.

  Wir definieren die Grad-0-Untergruppe $\Div^0(C)\subset\Div(C)$ als
  \begin{gather*}
    \Div^0(C) \coloneqq \left\{ D\in\Div(C) \;\middle|\; \deg D=0 \right\}
  \end{gather*}
  und entsprechend $\Pic^0(C)$ über die exakte Sequenz
  \begin{center}
    \begin{tikzcd}
      1 \arrow[r] &\algK^* \arrow[r] &\algK(C)^* \arrow[r, "\div"]
      &\Div^0(C) \arrow[r] &\Pic^0(C) \arrow[r] &0
    \end{tikzcd}
  \end{center}
\end{Definition}

Aus der Algebraischern Geometrie ist bekannt, dass die Punkte einer
Elliptischen Kurve $E$ mit einer geometrischen Addition eine Gruppe
bilden.

\begin{Satz}
  Für Elliptische Kurven (d.\,h. Kurven vom Genus 1 mit oben genannter
  Gruppenstruktur zum neutralen Element $O\in E$)gibt es einen
  Gruppenisomorphismus
  \begin{align*}
    \sigma\colon \Pic^0(E) &\longto E\\
    \overline{(D)} &\mapsto \overline{(P)}
                     \quad\text{mit}\quad D\sim (P)-(O)\\
    \overline{(P)-(O)} &\longmapsfrom P : \tau
  \end{align*}
\end{Satz}
Wir werden die beiden Gruppen im Folgenden immer miteinander
identifizieren.

\begin{Satz}
  Substraktion und Addition bzw. Translation auf Elliptischen Kurven
  sind Morphismen von Kurven. Die Abbildungen sind ($Q\in E$):
  \begin{align*}
    \bullet+Q\colon E\times E&\longto E  
    &&\text{und}  
    &-\bullet\colon E&\longto E \\
    P&\longmapsto P+Q             
    &&&P&\longmapsto -P
  \end{align*}
  \begin{proof}
    \cite[siehe][Theorem 3.6]{silverman}
  \end{proof}
\end{Satz}


%---

\section{Definition, Basic Properties}
Wir werden unsere Betrachtung von Morphismen nun auf sogenannte
Isogenien einschränken und sehen, dass diese Gruppenhomomorphismen
sind.

\begin{Definition}[Isogenie]
  Ein Morphismus $\phi\colon E_1\to E_2$ von Elliptischen Kurven wird
  Isogenie genannt, wenn $\phi(O)=O$.
  Gibt es solch ein $\phi$, so heißen $E_1$ und $E_2$ isogen.
\end{Definition}

\begin{Definition}
  Wir erhalten die Kategorie der Elliptischen Kurven mit Isogenien als
  Morphismen und entsprechend für $E$, $E_1$, $E_2$ Elliptische Kurven
  \begin{alignat*}{2}
    \Hom(E_1, E_2)&=\{\text{Isogenien }\phi\colon E_1\to E_2\}
    \qquad&&\text{Morphismengruppe}
    \\
    \End(E)&=\Hom(E,E)
    &&\text{Endomorphismenring}\\
    \Aut(E)&=\End(E)^*
    &&\text{Automorphismengruppe} 
  \end{alignat*}
  $\Hom(E_1, E_2)$ hat eine Gruppenstruktur mit Addition
  $(\phi+\psi)(P)\coloneqq \phi(P)+\psi(P)$ 
  und $\End(E)$ wird mit Morphismenverknüpfung als Multiplikation zu 
  einem Ring, dessen multiplikative Gruppe $\Aut(E)$ ist.
\end{Definition}

\begin{Bemerkung}
Wir können jeden Morphismus $F\colon E_1\to E_2$ von Elliptischen
Kurven, der den $O$-Punkt nicht erhält, auf eine Isogenie
zurückführen.
Denn nach \autoref{additionmorphismus} ist die Translation
$\tau_{-F(O)}\colon E_2\to E_2$ um $-F(O)\in E_2$
ein Morphismus, also auch 
\begin{gather*}
  F'\coloneqq\tau_{-F(O)}\circ F\colon E_1\to E_2
\end{gather*}
$F'$ ist eine Isogenie und $F=\tau_{F(O)}\circ F'$ kann über
eine Translation verknüpft mit dieser Isogenie dargestellt werden.
\end{Bemerkung}

\section{m-Torsions-Untergrupppe}

\begin{Beispiel}
  Nach \autoref{additionmorphismus} ist für eine Elliptische Kurve $E$
  und $m\in\Z$ die \enquote{Multiplikation mit $m$}
  \begin{align*}
    [m]\colon E &\longto E\\
    P&\longmapsto [m](P) \coloneqq
       \underbrace{P+P+\dotsb+P}_{m\text{-mal}}
  \end{align*}
  ein Morphismus und, da $[m](O)=O$, eine Isogenie.
  Zu Elliptische Kurven über Körpern von Charakteristik 0, deren
  Endomorphismen nicht allein diese Multiplikationen sind, sagt man,
  sie haben komplexe Multiplikation.
\end{Beispiel}

\begin{Lemma}
  Für eine Elliptische Kurve $E$ und $m\in\Z\setminus\{0\}$
  ist $[m]\colon E\to E$ nicht konstant, d.\,h. $[m]\neq[0]$.
  \begin{proof}
    \cite[siehe][Proposition 4.2 (a)]{silverman}
  \end{proof}
\end{Lemma}

\begin{Korollar}
  Für Elliptische Kurven $E_1$, $E_2$ ist $\Hom(E_1,E_2)$ ein
  torsionsfreier $\Z$-Modul, d.\,h. für jede Isogenie
  $\phi\in\Hom(E_1,E_2)$ ist $[m]\circ\phi\neq[0]$ für $m\in\Z\setminus\{0\}$.
    \begin{proof}
    \cite[siehe][Proposition 4.2 (b)]{silverman}
  \end{proof}
\end{Korollar}

\begin{Definition}[$m$-Torsionsuntergruppe]
  Für einen Elliptische Kurve $E$ und $m\in\Z\setminus\{0\}$ ist die
  $m$-Torsionsuntergruppe von $E$
  \begin{gather*}
    E[m] \coloneqq \left\{ P\in E\;\middle|\; [m](P)=O\right\}
    = E[-m]
  \end{gather*}
  Die Torsionsuntergruppe von $E$ ist die Vereinigung
  \begin{gather*}
    E_\text{tors} \coloneqq \bigcup_{\mathclap{m=1}}^{\infty}E[m]
  \end{gather*}
\end{Definition}



% ---

\section{Isogenien sind Homomorphismen}
\begin{Satz}\label{isogenienhoms}
  Isogenien sind Homomorphismen zwischen den Punktgruppen von
  Elliptischen Kurven, d.\,h. für eine Isogenie $\phi\colon E_1\to
  E_2$ gilt
  \begin{gather*}
    \phi(P+Q) = \phi(P) + \phi(Q)
    \qquad \text{für alle } P,Q\in E_1
  \end{gather*}
\begin{proof}
  Für die konstante Abbildung $\phi\equiv O$ ist es klar.
  Ist $\phi$ nicht konstant, machen wir uns die Gruppenisomorphie
  zwischen der Grad-0-Untergruppe der Picard-Gruppe und der
  Punktegruppe einer Elliptischen Kurve zunutze
  (s.\,\autoref{gruppenaequivalenz}).
  Wir erhalten 
  aus \autoref{gruppenaequivalenz}\,i) die Gruppenisomorphismen 
  \begin{alignat*}{3}
    &\kappa_1\colon&
    E_1&\overset\sim\longto \Pic^0(E_1)
    &P &\longmapsto \overline{(P)-(O)}\\
    &\kappa_2\colon&
    E_2&\overset\sim\longto \Pic^0(E_2)
    &P &\longmapsto \overline{(P)-(O)}\\
  \end{alignat*}
  und aus \autoref{gruppenaequivalenz}\,ii) den zu $\phi$
  korrespondierenden Gruppenhomomorphismus auf den Grad-0-Untergruppen
  der Picard-Gruppen
  \begin{align*}
    \phi_*\colon \Pic^0(E_1) 
    &\longto \Pic^0(E_2)\\
    \overline{\sum_{\mathclap{P\in E_1}}n_P\cdot(P)}
    &\longmapsto 
      \overline{\sum_{\mathclap{P\in E_1}}n_P\cdot(\phi(P))}
      \qquad .
  \end{align*}
  Insgesamt bekommen wir das kommutative Diagramm
  \begin{center}
    \begin{tikzcd}
      E_1 
      \arrow[r, "{\kappa_1}"{below}, "\sim"{above}]
      \arrow[d,"\phi"{left}] 
      & \Pic^0(E_1) 
      \arrow[d, "\phi_*"{right}]
      \arrow[dl, phantom, "\circlearrowleft"{description}]
      \\
      E_2 
      \arrow[r, "{\kappa_2}"{below}, "\sim"{above}]
      & \Pic^0(E_2) 
    \end{tikzcd}
  \end{center}
  welches zeigt, dass $\phi=\kappa_2^{-1}\circ\phi_*\circ\kappa_1$
  ist, also ein Gruppenhomomorphismus.
\end{proof}
\end{Satz}

%---

\section{Galois-Theorie elliptischer Funktionenkörper}
Aus \autoref{funktionenkoerper} wissen wir bereits, dass Isogenien
durch eine Kategorienantiäquivalenz als Körpererweiterungen aufgefasst
werden können und entsprechend für $\phi\colon E_1\to E_2$ der
Erweiterungsgrad $\deg(\phi)$, der Separabilitätsgrad $\degs(\phi)$
und der Inseparabilitätsgrad $\degi(\phi)$ definiert wird.
Im Folgenden sehen wir, inwiefern uns diese Charakterisierung von
Nutzen sein kann.

\begin{Satz}[Galois Theory elliptischer Funktionenkörper]\label{galois}
  Sei $\phi\colon E_1\to E_2$ eine nicht-konstante Isogenie.
  \begin{enumerate}[label=\roman*)]
  \item Für alle $P_2\in E_2$ gilt
    \begin{gather*}
      \#\phi^{-1}(P_2) = \degs(\phi)
    \end{gather*}
  \item Für alle $P_1\in E_1$ gilt
    \begin{gather*}
      \#e_{\phi}(P_1) = \degi(\phi)
    \end{gather*}
  \item Sei für einen Punkt $T\in E_1$ wieder 
    $\tau_T\colon P\mapsto P+T$ die Translationsabbildung und 
    $\tau_T^*$ die induzierte Abbildung auf $\algK(E_1)$.
    Dann ist
    \begin{align*}
      \Omega\colon
      \ker(\phi) 
      &\longto \Aut_{\phi^*(\algK(E_2))}\left( \algK(E_1) \right) \\
      T 
      &\longmapsto \left( \tau_T^* \colon f\mapsto f\circ \tau_T \right)
    \end{align*}
    ein Gruppenisomorphismus.
  \item Ist $\phi$ separabel, 
    so ist $\phi$ unverzweigt,
    (d.\,h. $e_\phi(P)=1$ für alle $P\in E_2$),
    $\algK(E_1)$ ist eine Galoiserweiterung von $\phi^*(\algK(E_2))$
    und
    \begin{gather*}
      \#\ker(\phi) = \deg(\phi)
    \end{gather*}

  \end{enumerate}

\begin{proof}~
  \begin{enumerate}[label=\roman*)]
  \item Für diesen Teil müssen wir nur die Aussage aus
    \autoref{sepgrad}\,ii) (die Behauptung in \emph{i)} von oben gilt
    für alle außer endlich viele Punkte in $E_2$) erweitern. Dazu
    zeigen wir, dass $\#\phi^{-1}(Q)$ für alle Punkte $Q\in E_2$
    gleich ist. 
    Betrachte also $Q,Q'\in E_2$. Nachdem $\phi$ surjektiv ist
    (s.\,\autoref{morphismensurj}), gibt es einen Punkt $R\in E_1$ mit
    \begin{gather*}
      \phi(R) = Q - Q'
    \end{gather*}
    Nutzen wir aus, dass $\phi$ als Isogenie ein Gruppenhomomorphismus
    zwischen den Punktegruppen von $E_1$ und $E_2$ ist nach
    \autoref{isogenienhoms}, erhalten wir, dass die Translationen
    \begin{align*}
      \phi^{-1}(Q) &\longto \phi^{-1}(Q')
      &\phi^{-1}(Q) &\longfrom \phi^{-1}(Q')\\
      P &\longmapsto P+R
      &P'-R &\longmapsfrom P'
    \end{align*}
    wohldefinierte, zueinander inverse Bijektionen sind.
    Denn für jedes $P\in\phi^{-1}(Q)$ ist
    $\phi(P+R)=\phi(P)+\phi(R)=Q'$, d.\,h. $(P+R)\in\phi^{-1}(Q')$,
    und umgekehrt mit $-R$.
    Damit braucht nur für einen Punkt $P\in E_2$ geben, für den das
    gilt, welcher durch den Beweis zu \autoref{sepgrad}\,ii) gegeben ist
    % \cite[siehe][II.6.8]{hartshorne}
    ??
    
  \item Hier benötigen wir die Multiplikativität des Verzweigungsindex
    aus \autoref{sepgrad}\,iii), um zu zeigen, dass auch alle Punkte
    in $\phi^{-1}(Q)$ den gleichen Verzweigungsindex bzgl. $\phi$ haben.
    Betrachte $P,P'\in E_1$ im Urbild $\phi^{-1}(Q)$. Das Bild ihrer
    Differenz $R\coloneqq P'-P$ ist folglich
    $\phi(P-P')=\phi(P)-\phi(P')=O$, weshalb die Verknüpfung
    $\psi\circ\tau_R = \psi$ mit der Translation um $R$ wieder $\psi$
    ist.
    Dann gilt für die Verzweigungsindizes aufgrund der
    Multiplikationsregel und mit $e_{\tau_R}\equiv1$ ($\tau_R$ ist
    Isomorphismus):
    \begin{gather*}
      e_\phi(P) 
      = e_{\phi\circ\tau_R}(P) 
      = e_{\tau_R}(P) \cdot e_\phi(\tau_R(P))
      = e_\phi(\tau_R(P))
      = e_\phi(P+R)
      = e_\phi(P')
    \end{gather*}
    Nachdem mit dieser Begründung jeder Punkt in $\phi^{-1}(Q)$
    denselben Verzweigungsindex hat, kann man folgendes Produkt
    umschreiben:
    \begin{align*}
      \degs(\phi) \cdot \degi(\phi)
      &= \deg(\phi)\\
      &\overset{\mathllap{\ref{sepgrad}\,i)}}{=}
        \sum_{\mathclap{P\in\phi^{-1}(Q)}} e_\phi(P)\\
      &\overset{\mathllap{\text{alle Pkte gleich}}}{=}
        \left( \#\phi^{-1}(Q) \right) \cdot e_\phi(P)\\
      &\overset{\mathllap{\text{\emph{i)}}}}{=}
        \degs(\phi)\cdot e_\phi(P)
    \end{align*}
    Kürzen liefert die Behauptung.
    \item 
      \begin{description}
        \item[Wohldefiniertheit] $\tau_T$ ist bereits als Morphismus
          von Kurven ein Automorphismus von $E_1$ und damit ist
          $\tau_T^*$ ein Automorphismus von $\algK(E_1)$. Bleibt zu
          zeigen, dass $\phi^*(\algK(E_2))$ fixiert wird.
          Sei $T\in\ker(\phi)$, dann ist $\phi\circ\tau_T=\phi$.
          Für $f\in\algK(E_2)$ erhalten wir
          \begin{gather*}
            \tau_T^*(\phi^*(f)) = (\phi\circ\tau_T)^*(f) = \phi^*(f)
          \end{gather*}
          Also wird $\phi^*(\algK(E_2))$ fixiert.
        \item[Homomorphismus] Seien $S,T\in E_1$. Laut Definition ist
          klar, dass $\tau_S\circ\tau_T = \tau_{S+T}$. Also
          \begin{align*}
            \Omega(T+S)(f)
            &= \tau^*_{T+S}(f)\\
            &= f\circ\tau_{T+S} 
            = f\circ\tau_{S+T}
            = f\circ\tau_{S}\circ\tau_{T}\\
            &= \tau^*_T\circ\tau^*_S(f)
            = \Omega(T)\circ\Omega(S)(f)
          \end{align*}
        \item[Surjektivität]
          Für eine vereinfachte Schreibweise identifizieren wir im
          Folgenden den Unterkörper ${\phi^*(\algK(E_2))}$ von
          $\algK(E_1)$ mit $\algK(E_2)$.
          Nach Algebra kann jeder
          $\algK(E_2)$-Automorphismus von $\algK(E_1)$
          erweitert werden zu einer 
          $\algK(E_2)$-Einbettung von $\algK(E_1)$ in den
          algebraischen Abschluss $\overline{\algK(C_2)}$. Laut 
          Galoistheorie gilt folgende Ungleichung
          \begin{align*}
            \#\Aut_{\algK(E_2)}\left( \algK(E_1) \right) 
            &\leq \#\Hom_K\left( \algK(C_1), \overline{\algK(C_2)} \right)\\
            &\eqqcolon
              \degs\left(\algK(C_1)/\algK(C_2)\right)\\
            &\eqqcolon \degs(\phi)
          \end{align*}
          Insgesamt gilt mit der Aussage $\degs(\phi)=\#\phi^{-1}(O)$
          aus \emph{i)}
          \begin{gather*}
            \#\Aut_{\algK(E_2)}\left( \algK(E_1) \right) 
            \leq \degs(\phi)
            \overset{\emph{i)}}{=} \#\phi^{-1}(O)
            =\#\ker(\phi)
          \end{gather*}
          Nachdem es sich um eine endliche Erweiterung handelt, sind
          alle Teile der Ungleichung endlich und wir erhalten die
          Surjektivität von $\Omega$ mit dem Schubladenprinzip.
        \item[Injektivität] Sei 
          $\tau_T^* = \Id_{\algK(E_1)} = \tau_O^*
          \in\Aut_{\phi^*(\algK(E_2))}( \algK(E_1) )$.
          Das heißt für alle Funktionen $f\in\algK(E_1)$
          \begin{gather*}
            f 
            = f\circ\tau_O
            = \tau_O^*(f) 
            = \tau_T^*(f) 
            = f\circ\tau_T
          \end{gather*}
          Aus der Definition der rationalen Abbildungen $\tau_T$ und
          $\tau_O$ \cite[siehe][Group Law Algorith 2.3(c)]{silverman},
          die stark von den Koordinaten von $T$ und $O$ abhängen,
          geht hervor, dass dann bereits die Koordinaten von
          $T$ und $O$ übereinstimmen müssen. Damit muss $T=O$ sein.
        \end{description}
    \item Sei $\phi$ separabel. Das heißt per Definition
      $\deg(\phi)=\degs(\phi)$.
      \begin{description}
        \item[Kernmächtigkeit]
          Mit \emph{i)} gilt
          $
          \#\ker(\phi)
          \coloneqq \#\phi^{-1}(O)
          \overset{\mathclap{\emph{i)}}}{=} \degs(\phi)
          \overset{{\text{sep.}}}{=} \deg(\phi)
          $.
        \item[Unverzweigtheit] 
          Entsprechend ist
          $\degi(\phi)=\frac{\deg(\phi)}{\degs(\phi)}=1$
          und mit \emph{ii)} gilt
          $e_\phi(P)=1$ für alle $P\in E_1$, d.\,h. $\phi$
          ist unverzweigt.
        \item[Galoiserweiterung]
          Mit \emph{i)} und \emph{iii)} gilt
          \begin{align*}
            \#\Aut_{\phi^*(\algK(E_2))}\left( \algK(E_1) \right) 
            \overset{\mathclap{\text{\emph{iii)}}}}{=}
              \#\ker(\phi)
            \overset{{\text{s.o.}}}{=} \deg(\phi)
            \coloneqq \left[\algK(C_1):\phi^*(\algK(C_2))\right]
          \end{align*}
          was äquivalent zu
          $\algK(C_1)/\phi^*(\algK(C_2))$ galoissch ist.
        \end{description}
      \end{enumerate}
\end{proof}
\end{Satz}

Der nächste Satz macht eine Aussage darüber, wie man Isogenien anhand
ihrer Kerne charakterisieren kann.
Zuvor benötigen wir noch ein Lemma, das uns zu einer Körpererweiterung
von Funktionenkörpern eine eindeutige Isogenie liefert.

\begin{Lemma}\label{erweiterungzuratabb}
  Sei $\iota\colon \K(C_2)\to\K(C_1)$ eine Körpererweiterung von
  Funktionenkörpern zu über $K$ definierten, glatten Kurven $C_1$ und
  $C_2$, die $K$ fixiert. Dann gibt es eine 
  eindeutige, über $K$ definierte, nicht-konstante, rationale
  Abbildung $\lambda\colon C_1\to C_2$ mit $\lambda^* = \iota$.
  Ist $C_1$ glatt, ist $\lambda$ nach \autoref{ratabbglattekurven}
  ein Morphimus.
  \begin{proof}
    \begin{description}
      \item[Existenz] Sei \OE{} $C_2\in\P^n$ nicht in der Hyperebene
        $X_0=0$ enthalten und
        $g_i\coloneqq\overline{\left(\frac{X_i}{X_0}\right)}\in\K(C_2)$.
        Dann erfüllt die Abbildung
        \begin{gather*}
          \lambda\coloneqq \left[1, \iota(g_1),\dotsc,\iota(g_n)\right]
        \end{gather*}
        die gewünschte Eigenschaft, denn auf den Erzeugern $g_i$ gilt
        \begin{gather*}
          \iota(g_i) 
          = \frac{\iota(g_i)}{1} 
          = \left(\frac{X_i}{X_0}\right)\circ\lambda
          = g_i\circ\lambda
          = \lambda^*(g_i)
        \end{gather*}
        Sie ist nicht konstant, da die
        $g_i$ nicht alle konstant sind und $\iota$ injektiv ist, also
        die $\iota(g_i)$ ebenfalls nicht alle konstant sind. Und sie
        ist offensichtlich rational sowie über $\K$ definiert.
      \item[Eindeutigkeit]
        Sei $\tlambda\coloneqq[f_0,\dotsc,f_n]$ eine weitere
        Abbildungen mit den gesuchten Eigenschaften.
        Dann gilt
        \begin{gather*}
          \frac{f_i}{f_0}
          %= \left(\frac{X_i}{X_0}\right)\circ\tlambda 
          = g_i\circ\tlambda
          = \tlambda^*(g_i) 
          \overset{\text{Annahme}}{=}
          \iota(g_i)
        \end{gather*}
        also 
        \begin{gather*}
          \tlambda
          \coloneqq \left[ f_0,f_1,\dotsc,f_n \right]
          = \left[ 1, \frac{f_1}{f_0}, \dotsc, \frac{f_n}{f_0}\right]
          = \left[ 1, \iota(g_1), \dotsc, \iota(g_n) \right]
          \eqqcolon \lambda
        \end{gather*}
    \end{description}
  \end{proof}
\end{Lemma}

\begin{Satz}
  Sei $\phi\colon E_1\to E_2$ eine separable Isogenie.
  Jede weitere Isogenie $\psi\colon E_1\to E_3$ mit
  \begin{gather*}
    \ker\phi \subset \ker\psi
  \end{gather*}
  faktorisiert eindeutig über $\psi$, d.\,h. es gibt eine eindeutige
  Isogenie $\lambda\colon E_2\to E_3$ mit
  \begin{gather*}
    \psi = \lambda\circ\phi\;.
  \end{gather*}
  \begin{proof}
    \autoref{galois}\,iv) liefert uns, dass
    $\algK(E_1)/\phi^*(\algK(E_2))$ eine Galoiserweiterung ist mit
    Galoisgruppe $\Aut_{\phi^*(\algK(E_2))}(\algK(E_1))$.
    Der Isomorphismus aus \autoref{galois}\,iii) zeigt, dass sie aus
    Elementen der Form $\tau_T^*$ mit $T\in\ker(\phi)$ besteht.
    Da nach Voraussetzung $\ker(\phi)\subset\ker(\psi)$ gilt,
    ist auch
    \begin{align*}
      \Aut_{\phi^*(\algK(E_2))}(\algK(E_1))
      &= \left\{ \tau_T^* \;\middle|\; T\in\ker(\phi) \right\} \\
      &\subset
      \left\{ \tau_T^* \;\middle|\; T\in\ker(\psi) \right\}
      = \Aut_{\psi^*(\algK(E_3))}(\algK(E_1))
    \end{align*}
    D.\,h. die Galoisgruppe fixiert den Unterkörper
    $\psi^*(\algK(E_3))$, weshalb er nach Galoistheorie ein
    Unterkörper von $\phi^*(\algK(E_2))$ sein muss.
    Wir erhalten folgende Kette von Körpererweiterungen
    \begin{gather*}
      \psi^*(\algK(E_3)) 
      \subset \phi^*(\algK(E_2)) 
      \subset \algK(E_1)
    \end{gather*}
    Da $\phi^*$ injektiv und $\phi^*|_{\im(\phi^*)}$ ein
    Körperisomorphismus ist, erhalten wir eine Körpererweiterung
    \begin{gather*}
      \iota\coloneqq 
      (\phi^*)^{-1} \circ \psi^*
      \colon \algK(E_3) \longto \algK(E_2)
    \end{gather*}
    \autoref{erweiterungzuratabb} liefert jetzt einen Morphismus
    $\lambda\colon E_3\to E_2$ mit
    \begin{gather*}
      \lambda^* = \iota = (\phi^*)^{-1} \circ \psi^*
      \qquad\text{bzw.}\qquad
      \phi^*\circ\lambda^* = \psi^*
    \end{gather*}
    also wie gesucht $\lambda\circ\phi = \psi$.
    Es bleibt zu zeigen, dass $\lambda$ eine Isogenie ist:
    \begin{gather*}
      \lambda(O) 
      \overset{\text{$\phi$ Isog.}}{=} \lambda(\phi(O))
      = \psi(O) 
      = O
    \end{gather*}
  \end{proof}
\end{Satz}

%---

\section{Eindeutige Faktorisierung}
\section{Modulorechnung von Elliptischen Kurven}

%\nocite{*}
\printbibliography
\end{document}
