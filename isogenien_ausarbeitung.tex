\documentclass[english, german, parskip=half]{scrartcl}
% \usepackage{mathpazo}
\usepackage{fontspec}
\setmainfont{Latin Modern Roman}
\usepackage[shorthands=off]{babel}
\usepackage{csquotes}

\usepackage[style=alphabetic, backend=biber]{biblatex}
\bibliography{isogenien_literatur.bib}

\usepackage{mathtools}
\usepackage{amssymb}
\usepackage{amsthm}
\usepackage{dsfont}
\usepackage{stmaryrd}
\usepackage{tikz-cd}
\usetikzlibrary{babel}
\hyphenation{wo-raus}

\usepackage{enumitem}

\usepackage[colorlinks=true, linkcolor=blue]{hyperref}
\hypersetup{
  pdfauthor={Gesina Schwalbe},
  pdftitle={Isogenien Seminarausarbeitung}
}

% DEFINITIONS
\newtheorem{Satz}{Satz}[section]
\newtheorem{Theorem}[Satz]{Theorem}
\newtheorem{Proposition}[Satz]{Proposition}
\newtheorem{Lemma}[Satz]{Lemma}
\newtheorem{Korollar}[Satz]{Korollar}
\theoremstyle{definition}
\newtheorem{Definition}[Satz]{Definition}
\newtheorem{LemmaDefinition}[Satz]{Lemma/Definition}
\newtheorem{Beispiel}[Satz]{Beispiel}
\theoremstyle{remark}
\newtheorem{Bemerkung}[Satz]{Bemerkung}

\newcommand*{\N}{\mathds{N}}
\newcommand*{\Z}{\mathds{Z}}
\newcommand*{\K}{\ensuremath{K}} % Körper
\newcommand*{\algK}{\ensuremath{\overline K}} % algebraischer Abschluss
\renewcommand*{\P}{\ensuremath{\mathds{P}}} % projektiver Raum
\newcommand*{\longto}{\longrightarrow}
\newcommand*{\longfrom}{\longleftarrow}
\newcommand*{\Xn}{\underline{X}} % Variablenmenge
\newcommand*{\degs}{\operatorname{\deg}_s} % Separabilitätsgrad
\newcommand*{\degi}{\operatorname{\deg}_i} % Inseparabilitätsgrad
\DeclareMathOperator{\ord}{ord} % diskrete Bewertung
\DeclareMathOperator{\Div}{Div} % Divisorengruppe
\DeclareMathOperator{\Pic}{Pic} % Picard Gruppe
\renewcommand{\div}{\operatorname{div}}
\renewcommand{\O}{O}
\DeclareMathOperator{\Char}{char} % Charakteristik
\DeclareMathOperator{\Hom}{Hom} % Homomorphismenmenge
\DeclareMathOperator{\End}{End} % Endomorphismenmenge
\DeclareMathOperator{\Aut}{Aut} % Automorphismenmenge
\newcommand{\Id}{\mathrm{id}} % Identitätsabbildung
\DeclareMathOperator{\im}{im} % image
\newcommand{\tlambda}{\tilde\lambda} % alternative Abbildung \lambda
\newcommand{\F}{F} % endliche Untergruppe
\DeclareMathOperator{\genus}{genus} % Kurvengeschlecht

% TITEL
\titlehead{Seminar Elliptische Kurven \\
  von Prof. Kerz,
  SS2016, Universität Regensburg}
\subject{Vortragsausarbeitung}
\title{Isogenien}
\author{Gesina Schwalbe}

\begin{document}
\maketitle
\tableofcontents

\section{Motivation}
In vorherigen Vorträgen wurden elliptische Kurven eingeführt, ihre
Eigenschaften als Varietäten sowie die Eigenschaften als Gruppen
bezüglich der geometrischen Addition ihrer Punkte.
Interessanterweise sind diese miteinander kompatibel in dem Sinne,
dass die Addition selbst eine rationale Abbildung zwischen Varietäten
ist.

Diese Ausarbeitung befasst sich aufbauend darauf mit der Frage,
inwiefern die Gruppenhomomorphismen zwischen den Punktgruppen
ebenfalls mit den Varietätseigenschaften kompatibel sind.
Dafür wird nach der Wiederholung mit der Definition sogenannter
Isogenien, bestimmter Morphismen zwischen elliptischen Kurven,
begonnen, welche sich als Gruppenhomomorphismen entpuppen werden.
Diese werden dann weiter untersucht auf die Eigenschaften ihrer Kerne,
der Anwendbarkeit des Homomorphiesatzes und Quotientenbildung.

\section{Wiederholung}
\subsection{Morphismen von Kurven}

\begin{Definition}[Definiertheit]
  Sei $C$ eine Kurve, $P\in C$ und $\algK[C]_P$ die Lokalisierung
  ihres affinen Koordinatenrings nach dem Maximalideal $M_P$.
  Wir bezeichnen mit
  \begin{align*}
    \ord_P\colon \algK[C]_P &\longto \N_0 \cup \{\infty\} \\
    f &\mapsto \max\left\{d\in\Z\;\middle|\; f\in M_P^d \right\}
  \end{align*}
  die diskrete Bewertung auf $\algK[C]_P$ bzw. durch kanonische
  Erweiterung
  \begin{align*}
    \ord_P\colon \algK(C) &\longto \Z \cup \{\infty\} \\
    \frac{f}{g} &\longmapsto \ord_P(f) - \ord_P(g)
  \end{align*}
  die diskrete Bewertung auf dem Quotientenkörper $\algK(C)$.

  Anschaulich gibt $\ord_P(f)$ für $f\in\algK(C)$ an, ob $f$
  eine Nullstelle ($\ord_P(f)>0$) oder eine Polstelle ($\ord_P(f)<0$)
  in $P$ hat und mit welcher Vielfachheit ($=|\ord_P(f)|$).
  Insbesondere ist $f$ in $P$ auswertbar, wir bezeichnen es als
  definiert, wenn $\ord_P(f)\geq 0$.

  Für eine allgemeine projektive Varietät $V$ heißt eine Funktion
  $f\in\algK(V)$ definiert in $P\in V$, falls $f\in\algK[V]_P$.
\end{Definition}

\begin{Definition}[Morphismus projektiver Varietäten]
  Für projektive Varietäten $V_1\subset\P^m$, $V_2\subset\P^n$ heißt
  $\phi\colon V_1\to V_2$ eine rationale Abbildung,
  falls $\phi$ in der Form
  \begin{align*}
    \phi = [f_0,\dotsc, f_n]\colon V_1 &\longto V_2\\
    P &\longmapsto \left[f_0(P), \dotsc, f_n(P)\right] 
        \qquad \text{mit } f_i\in\algK(V_1)
  \end{align*}
  geschrieben werden kann, 
  wobei für jeden Punkt $P\in V_1$, 
  in dem alle $f_i$ definiert sind, % f_i in P auswertbar
  $(f_0(P),\dotsc,f_n(P))\neq0$. % Wohldefiniertheit
  Solch ein $\phi$ heißt definiert über $\K$,
  falls $f_i\in\K(V_1)$. 
  % Beachte hierbei die Invarianz
  % bzgl. Multiplikation mit Einheiten $\lambda$, denn
  % $[f_0,\dotsc,f_n]=[\lambda f_0,\dotsc,\lambda f_n]\in\P^n$.

  Eine rationale Abbildung $\phi$ heißt regulär in $P\in V_1$,
  falls es ein $g\in\algK(V)$ gibt, sodass
  alle $gf_i$ in $P$ definiert sind 
  und $(gf_0(P),\dotsc,gf_n(P))\neq0$.
  Beachte hierbei, 
  dass für ein in $P$ definiertes $g\in\algK(V)$ mit $g(P)\neq0$ gilt
  $[f_0,\dotsc,f_n]=[gf_0,\dotsc, gf_n]\in\P^n$
  (denn $[x_0,\dotsc,x_n]=[\lambda x_0,\dotsc,\lambda x_n]\in\P^n$ für
  $\lambda\in\algK$).
  
  Ein Morphismus von projektiven Varietäten ist eine rationale
  Abbildung $\phi\colon V_1\to V_2$, die in allen $P\in V_1$ regulär
  ist.
\end{Definition}

\begin{Lemma}\label{ratabbglattekurven}
  Sei $C\subset\P^2$ eine Kurve und $V\subset\P^n$ eine projektive
  Varietät. 
  Eine rationale Abbildung $\phi=[f_0,\dotsc, f_n]\colon C\to V$ ist
  regulär in allen regulären Punkten von $C$.
  Insbesondere sind rationale Abbildungen von glatten Kurven
  Morphismen.
  \begin{proof}[Beweisskizze]
    Für einen regulären Punkt $P\in C$ und einen Erzeuger
    $t\in\algK(C)$ von $M_P$ gilt mit 
    $r\coloneqq\min_{0\leq i\leq n}\{\ord_Pf_i\}$,
    dass $[t^{-r}f_0,\dotsc,t^{-r}f_n]$ regulär in $P$ ist.
    (Verwende die Tatsache, dass $g\in C$ genau dann regulär in $P$
    ist, wenn $\ord_P(g)\geq0$.)
  \end{proof}
\end{Lemma}

\begin{Lemma}\label{morphismensurj}
  Ein Morphismus von Kurven ist entweder konstant oder surjektiv.
  \begin{proof}
    \cite[siehe][Theorem II.2.3]{silverman}
  \end{proof}
\end{Lemma}

% ---

\subsection{Kategorie der glatten Kurven und Kategorienäquivalenz}
\begin{Bemerkung}[Kategorie der glatten Kurven]
  Wir erhalten die Kategorie der glatten Kurven
  mit nicht-konstanten, rationalen Abbildungen als Morphismen 
  (nach Lemma \ref{ratabbglattekurven} und \ref{morphismensurj} also
  surjektive Morphismen von Kurven)
\end{Bemerkung}

\begin{LemmaDefinition}\label{funktionenkoerper}
  Eine rationale Abbildung $\phi\colon C_1\to C_2$ von auf $\K$
  definierten Kurven induziert eine Körpererweiterung über $\K$ auf
  den Funktionenkörpern
  \begin{align*}
    \phi^* \colon \K(C_2) &\longto \K(C_1)\\
    f &\longmapsto f\circ \phi
  \end{align*}
  Diese Erweiterung ist endlich
  \cite[siehe][Theorem II.2.4 (a)]{silverman}.

  Man kann $\phi$ anhand der Eigenschaften der Körpererweiterung
  $\K(C_1)/\phi^*(\K(C_2))$ charakterisieren und erhält
  dementsprechend für $\phi$ nicht-konstant
  \begin{alignat*}{2}
    \deg(\phi) &\coloneqq \left[ \K(C_1) : \phi^*(\K(C_2)) \right]
    \qquad&&\text{(Erweiterungs-)Grad von }\phi\\
    \degs(\phi) &\coloneqq \left[ \K(C_1) : \phi^*(\K(C_2)) \right]_s
    \qquad&&\text{Separabilitätsgrad von }\phi\\
    \degi(\phi) &\coloneqq \left[ \K(C_1) : \phi^*(\K(C_2)) \right]_i
    \qquad&&\text{Inseparabilitätsgrad von }\phi
  \end{alignat*}
  bzw. die Bezeichnungen endlich/separabel/inseparabel/rein
  inseparabel. Zur Erinnerung: Es gelten die Zusammenhänge
  \begin{align*}
    \degs(\phi) 
    &\coloneqq \left[ \K(C_1) : \phi^*(\K(C_2)) \right]_s
    \coloneqq \#\Hom_K(\K(C_1), \overline{\K(C_2)})\\
    \deg(\phi) &= \degs(\phi)\cdot\degi(\phi)
  \end{align*}
  Wir setzen $\deg\phi\coloneqq 0$ für konstantes $\phi$.
\end{LemmaDefinition}

\begin{Lemma}\label{erweiterungzuratabb}
  Sei $\iota\colon \K(C_2)\to\K(C_1)$ eine Körpererweiterung von
  Funktionenkörpern zu über $K$ definierten, glatten Kurven $C_1$ und
  $C_2$, die $K$ fixiert. Dann gibt es eine 
  eindeutige, über $K$ definierte, nicht-konstante, rationale
  Abbildung $\lambda\colon C_1\to C_2$ mit $\lambda^* = \iota$.
  Ist $C_1$ glatt, ist $\lambda$ nach \autoref{ratabbglattekurven}
  ein Morphimus.
  \begin{proof}
    \begin{description}
      \item[Existenz] Sei \OE{} $C_2\in\P^n$ nicht in der Hyperebene
        $X_0=0$ enthalten und
        $g_i\coloneqq\overline{\left(\frac{X_i}{X_0}\right)}\in\K(C_2)$.
        Dann erfüllt die Abbildung
        \begin{gather*}
          \lambda\coloneqq \left[1, \iota(g_1),\dotsc,\iota(g_n)\right]
        \end{gather*}
        die gewünschte Eigenschaft, denn auf den Erzeugern $g_i$ gilt
        \begin{gather*}
          \iota(g_i) 
          = \frac{\iota(g_i)}{1} 
          = \left(\frac{X_i}{X_0}\right)\circ\lambda
          = g_i\circ\lambda
          = \lambda^*(g_i)
        \end{gather*}
        Sie ist nicht konstant, da die
        $g_i$ nicht alle konstant sind und $\iota$ injektiv ist, also
        die $\iota(g_i)$ ebenfalls nicht alle konstant sind. Und sie
        ist offensichtlich rational sowie über $\K$ definiert.
      \item[Eindeutigkeit]
        Sei $\tlambda\coloneqq[f_0,\dotsc,f_n]$ eine weitere
        Abbildungen mit den gesuchten Eigenschaften.
        Dann gilt
        \begin{gather*}
          \frac{f_i}{f_0}
          %= \left(\frac{X_i}{X_0}\right)\circ\tlambda 
          = g_i\circ\tlambda
          = \tlambda^*(g_i) 
          \overset{\text{Annahme}}{=}
          \iota(g_i)
        \end{gather*}
        also 
        \begin{gather*}
          \tlambda
          \coloneqq \left[ f_0,f_1,\dotsc,f_n \right]
          = \left[ 1, \frac{f_1}{f_0}, \dotsc, \frac{f_n}{f_0}\right]
          = \left[ 1, \iota(g_1), \dotsc, \iota(g_n) \right]
          \eqqcolon \lambda
        \end{gather*}
    \end{description}
  \end{proof}
\end{Lemma}

\begin{Satz}[Kategorienäquivalenz]\label{kategorienaequivalenz}
  Die Aussagen von oben lassen sich zu einer Kategorienantiäquivalenz
  zwischen der Kategorie der glatten Kurven und der Kategorie der
  Körpererweiterungen $\K'$ über $\K$ mit Transzendenzgrad 1 und
  $\K'\cap\algK=\K$ verallgemeinern:
  \begin{align*}
    C/K &\Longrightarrow \K(C)\\
    \phi\colon C_1\to C_2 &\Longrightarrow \phi^*\colon\K(C_2)\to\K(C_1)
  \end{align*}
  \cite[siehe][Corollary I.6.11]{hartshorne} und
  \cite[siehe][Remark II.2.5]{silverman}.
  %% Körpererweiterungszeugs: siehe S.24–25
\end{Satz}

%---

\subsection{Verzweigungsindex}
\begin{Definition}[Verzweigungsindex]\label{verzweigungsindex}
  Sei $\phi\colon C_1\to C_2$ eine nicht-konstante Abbildung zwischen
  glatten Kurven.
  Dann ist der Verzweigungsindex von $\phi$ in einem Punkt
  $P\in C_1$ definiert als
  \begin{gather*}
    e_\phi(P) = \ord_P \left( \phi^*(t_{\phi(P)}) \right) \geq 1
  \end{gather*}
  wobei $t_{\phi(P)}$ ein Erzeuger von $M_{\phi(P)}$ ist (d.\,h. auch
  eindeutig bis auf Multiplikation mit Einheiten, da $P$ regulär ist).
\end{Definition}

\begin{Satz}\label{sepgrad}
  Sei $\phi\colon C_1\to C_2$ wieder eine nicht-konstante Abbildung
  zwischen glatten Kurven. 
  \begin{enumerate}[label=\roman*)]
    \item Es gilt für alle $Q\in C_2$
      \begin{gather*}
        \sum_{\mathclap{P\in\phi^{-1}(Q)}} e_\phi(P) = \deg\phi
      \end{gather*}
      Insbesondere gilt, da $\deg\phi$ endlich ist
      (s. \ref{funktionenkoerper}) und $e_\phi(P)\geq 1$,
      dass das Urbild $\phi^{-1}(Q)$ jedes Punkts $Q\in C_2$
      endlich viele Punkte in $C_1$ enthält.
    \item Für alle außer endlich viele Punkte $Q\in C_2$ gilt
      \begin{gather*}
        \#\phi^{-1}(Q) = \degs(\phi)
      \end{gather*}
    \item Der Verzweigungsindex ist multiplikativ
      in folgendem Sinne
      \begin{gather*}
        e_{\psi\circ\phi}(P) = e_\phi(P) \cdot e_\psi(\phi(P))
      \end{gather*}
    \end{enumerate}
    \begin{proof}
      \cite[siehe][Proposition II.2.6]{silverman}
      
      % zu \emph{ii)}: Nach Definition des Separabilitätsgrades gilt
      % (wir identifizieren $\phi^*(\algK(C_2))$ mit $\algK(C_2)$)
      % \begin{gather*}
      %   \degs(\phi) 
      %   \coloneqq \degs\left( \algK(C_1)/\algK(C_2) \right)
      %     \coloneqq
      %   \#\Hom_{\algK(C_2)}\left( \algK(C_1), \overline{\algK(C_2)} \right)
      % \end{gather*}
      % Mit unserer Kategorienantiäquivalenz aus
      % \autoref{kategorienaequivalenz} erhalten wir zur
      % $\algK$-Erweiterung $\overline{\algK(C_2)}$ eine glatte Kurve
      % $C_0$ mit einem Morphismus $\lambda\colon C_0\to C_2$, für das
      % $\lambda^*$ die Erweiterung $\algK(C_2)\to\overline{\algK(C_2)}$
      % ist (s. \autoref{erweiterungzuratabb}).
      % $\Hom_{\algK(C_2)}(\algK(C_1), \overline{\algK(C_2)})$
      % wird zur Menge der Homomorphismen $\psi$, die über $\phi$ wie
      % folgt faktorisieren:
      % \begin{center}
      %   \begin{tikzcd}
      %     C_0
      %     \arrow[r, "\psi"{below}]
      %     \arrow[rr, "\lambda"{below}, bend left=30]
      %     & C_1
      %     \arrow[r, "\phi"{below}]
      %     & C_2
      %     \\
      %     \algK(C_0)=\overline{\algK(C_2)}
      %     \arrow[from=r, "\psi^*"{above}]
      %     \arrow[from=rr, "\lambda^*"{above}, bend left=30]
      %     & \algK(C_1)
      %     \arrow[from=r, "\phi^*"{above}]
      %     & \algK(C_2)
      %   \end{tikzcd}
      % \end{center}
      % Sei $P\in C_2$. Dann 

    \end{proof}


\end{Satz}

% ---

\subsection{Divisorengruppe}
\begin{Definition}[Divisorengruppe]
  Für eine Kurve $C$ wird die von den Punkten von $C$ erzeugte, freie
  abelsche Gruppe mit $\Div(C)$ bezeichnet und ihre
  Elemente heißen Divisoren. Ein Punkt $P$ erhält als Divisor die
  Schreibweise $(P)$. 
  Der Grad eines Divisors $D=\sum_{P\in C}n_P\cdot(P)$ ist definiert als
  \begin{gather*}
    \deg D \coloneqq \sum_{P\in C} n_P
  \end{gather*}
  
  Ist $C$ glatt, definieren wir für eine Funktion
  $f\in\algK(C)^*$ den zugehörigen Divisor $\div(f)$
  als
  \begin{gather*}
    \div(f) \coloneqq \sum_{P\in C}\ord_P(f)\cdot(P)
  \end{gather*}
  $\div\colon\algK(C)^*\to\Div(C)$ ist ein Gruppenhomomorphismus,
  d.\,h. $\div(f\cdot g)=\div(f)+\div(g)$,
  mit Kern $\algK^*$.
  Ein Element im Bild von $\div$ wird prinzipal genannt.
  
  Die Picard-Gruppe $\Pic(C)$ von $C$ ist der Quotient
  $\Div(C)/\div(\algK(C)^*)$ der Divisorengruppe über der Untergruppe
  der prinzipalen Elemente. Wir schreiben 
  $(P)\sim(Q):\Leftrightarrow (P)-(Q)=\div(f)$ für ein $f\in\algK(C)^*$.

  Wir definieren die Grad-0-Untergruppe $\Div^0(C)\subset\Div(C)$ als
  \begin{gather*}
    \Div^0(C) \coloneqq \left\{ D\in\Div(C) \;\middle|\; \deg D=0 \right\}
  \end{gather*}
  und entsprechend $\Pic^0(C)$ über die exakte Sequenz
  \begin{center}
    \begin{tikzcd}
      1 \arrow[r] &\algK^* \arrow[r] &\algK(C)^* \arrow[r, "\div"]
      &\Div^0(C) \arrow[r] &\Pic^0(C) \arrow[r] &0
    \end{tikzcd}
  \end{center}
\end{Definition}

Aus der Algebraischen Geometrie ist bekannt, dass die Punkte einer
Elliptischen Kurve $E$ mit einer geometrischen Addition eine Gruppe
bilden.

\begin{Satz}\label{gruppenaequivalenz}
  Eine Elliptische Kurve sei als Gruppe ihrer Punkte aufgefasst
  mit der geometrischen Addition.
  \begin{enumerate}[label=\roman*)]
  \item 
    Für eine Elliptische Kurve $E$ mit neutralem Element $O\in E$ gibt
    es einen Gruppenisomorphismus
    \begin{align*}
      \sigma\colon \Pic^0(E) &\longto E\\
      \overline{(D)} &\mapsto P
                       \quad\text{mit}\quad D\sim (P)-(O)\\
      \overline{(P)-(O)} &\longmapsfrom P : \kappa
    \end{align*}

  \item
    Jeder nicht-konstante Morphismus $\phi\colon E_1\to E_2$, 
    auf elliptischen Kurven $E_1$, $E_2$ induziert Gruppenhomomorphismen
    \begin{alignat*}{3}
      &\phi_*\colon& \Div(E_1) 
      &\longto \Div(E_2),
      \qquad
      &(P) &\longmapsto (\phi(P))\\
      &\phi_*\colon& \Div^0(E_1) 
      &\longto \Div^0(E_2),
      &(P) &\longmapsto (\phi(P))\\
      &\phi_*\colon& \Pic^0(E_1) 
      &\longto \Pic^0(E_2),
      &\overline{(P)} &\longmapsto \overline{(\phi(P))}\\
    \end{alignat*}
  \end{enumerate}
  \begin{proof}
    Zu \emph{i)} \cite[siehe][Proposition III.3.4]{silverman} und
    zu \emph{ii)} \cite[siehe][Remark II.3.7]{silverman}.
    \end{proof}
  \end{Satz}
  Wir werden die beiden Gruppen im Folgenden immer miteinander
  identifizieren.


  \begin{Satz}\label{additionmorphismus}
    Substraktion und Addition (bzw. Translation $\tau_Q$ um $Q\in E$)auf
    Elliptischen Kurven sind Morphismen von Kurven. Die Abbildungen sind:
    \begin{align*}
      \tau_Q\coloneqq\bullet+Q\colon E\times E&\longto E  
      &&\text{und}  
      &-\bullet\colon E&\longto E \\
      P&\longmapsto P+Q             
      &&&P&\longmapsto -P
    \end{align*}
    \begin{proof}
      \cite[siehe][Theorem III.3.6]{silverman}
    \end{proof}
  \end{Satz}


  Eine weiterer wichtiger Zusammenhang ist die Hurwitz'sche Formel für
  das Geschlecht von Kurven.

  \begin{Satz}[Hurwitz]\label{hurwitz}
    Für einen separablen Morphismus $\phi\colon C_1\to C_2$ von
    glatten Kurven 
    je mit Geschlecht $g_i$ gilt
    \begin{gather*}
      2g_1-2
      \geq
      \deg(\phi)\cdot(2g_2-2) + \sum_{\mathclap{P\in C_1}}(e_P(\phi)-1)
    \end{gather*}
    über einem Körper $\K$ mit 
    $\Char(\K)\nmid e_P(\phi)$ für alle $P\in C_1$ 
    gilt Gleichheit.
    \begin{proof}
      \cite[siehe][Theorem II.5.9]{silverman}
    \end{proof}
  \end{Satz}


  %%% Local Variables:
  %%% mode: latex
  %%% TeX-master: "isogenien_ausarbeitung"
  %%% End:


% ---

\section{Kategorie der elliptischen Kurven}
Wir werden unsere Betrachtung von Morphismen nun auf sogenannte
Isogenien einschränken. Zusammen mit ihnen formen die elliptischen
Kurven über einem algebraisch abgeschlossenen Körper $\algK$ eine
Unterkategorie der glatten Kurven über $\algK$.
Deren Homomorphismenmengen tragen nicht nur eine von der Punktgruppe
induzierte Gruppenstruktur, sondern sind selbst torsionsfreie
$\Z$-Moduln, weshalb wir Torsions-Untergruppen definieren können.


\subsection{Isogenien}
\begin{Definition}[Isogenie]
  Ein Morphismus $\phi\colon E_1\to E_2$ von elliptischen Kurven wird
  Isogenie genannt, wenn $\phi(\O)=\O$.
  Gibt es solch ein $\phi$, so heißen $E_1$ und $E_2$ isogen.
\end{Definition}

\begin{Definition}
  Wir erhalten die Kategorie der elliptischen Kurven mit Isogenien als
  Morphismen und entsprechend für $E_1$, $E_2$, $E$ elliptische Kurven
  \begin{alignat*}{2}
    \Hom(E_1, E_2)&=\left\{\text{Isogenien }\phi\colon E_1\to E_2 \right\}
    \qquad&&\text{Morphismengruppe}
    \\
    \End(E)&=\Hom(E,E)
    &&\text{Endomorphismenring}\\
    \Aut(E)&=\End(E)^*
    &&\text{Automorphismengruppe} 
  \end{alignat*}
  $\Hom(E_1, E_2)$ hat eine Gruppenstruktur mit Addition
  $(\phi+\psi)(P)\coloneqq \phi(P)+\psi(P)$ 
  und $\End(E)$ wird mit Morphismenverknüpfung als Multiplikation zu 
  einem Ring, dessen multiplikative Gruppe $\Aut(E)$ ist.
\end{Definition}

\begin{Bemerkung}
  Wir können jeden Morphismus $\gamma\colon E_1\to E_2$ von elliptischen
  Kurven, selbst wenn er den $\O$-Punkt nicht erhält, auf eine Isogenie
  zurückführen.
  Denn nach \autoref{additionmorphismus} ist die Translation
  $\tau_{-\gamma(\O)}\colon E_2\to E_2$ um $-\gamma(\O)\in E_2$
  ein Morphismus, also auch 
  \begin{gather*}
    \widetilde{\gamma}\coloneqq\tau_{-\gamma(\O)}\circ \gamma\colon E_1\to E_2
  \end{gather*}
  $\widetilde{\gamma}$ ist eine Isogenie und $\gamma=\tau_{\gamma(\O)}\circ \widetilde{\gamma}$ kann über
  eine Translation verknüpft mit dieser Isogenie dargestellt werden.
\end{Bemerkung}

\subsection{Homomorphismengruppen und Torsions-Untergruppen}
Vorweg als Beispiel eine wichtige Form von Isogenien im
Endomorphismenring, die häufig bereits den gesamten Ring ausmachen.
\begin{Beispiel}[Multiplikationsisogenien]
  Nach \autoref{additionmorphismus} ist für eine elliptische Kurve $E$
  und $m\in\Z$ die \enquote{Multiplikation mit $m$}
  \begin{align*}
    [m]\colon E &\longto E\\
    P&\longmapsto [m](P) \coloneqq
       \underbrace{P+P+\dotsb+P}_{m\text{-mal}}
  \end{align*}
  ein Morphismus und wegen $[m](\O)=\O$ auch eine Isogenie.
  Hat eine elliptische Kurve über einem Körper von Charakteristik Null
  mehr Endomorphismen als nur diese Multiplikationen, sagt man,
  sie hat komplexe Multiplikation.
\end{Beispiel}

Eine wichtige Eigenschaft der Multiplikationsisogenien ist, dass sie
nicht konstant sind. Das liefert interessante Aussagen zur Struktur
der Homomorphismengruppe~und des Endomorphismenrings.
\begin{Lemma}\label{additionnichtkonstant}
  Für eine elliptische Kurve $E$ und $m\in\Z\setminus\{0\}$
  ist $[m]\colon E\to E$ nicht konstant, d.\,h. $[m]\neq[0]$.
  \begin{proof}[Beweisskizze]
    \cite[siehe][Proposition III.4.2 (a)]{silverman}
    
    Für $[2]\neq[0]$ nutzt man, dass nach der Duplikationsformel 
    \cite[Group Law Algorithm III.2.3(d)]{silverman} ein Punkt $P=(x,y)\in
    E$ mit $P+P=\O$ eine Gleichung der Form
    \begin{gather*}
      4x^3 + b_2x^2 + 2b_4 + b_6 = 0
    \end{gather*}
    ($b_i$ Koeffizienten der Weierstrassgleichung von $E$)
    erfüllen muss.
    Für $\Char(\algK)\neq 2$ ist dies eine polynomiale Gleichung und hat
    entsprechend nur endlich viele Lösungen. Nachdem $E$ unendlich
    viele Punkte hat
    \cite[siehe][Exercise I.4.8 oder Section I.6]{hartshorne}
    kann also $[2]$ für $\Char(\algK)\neq 2$ nicht konstant sein.

    Für $\Char(\algK)=2$ vereinfacht sich obige Gleichung zu
    \begin{gather*}
      b_2x^2 + b_6 = 0\,.
    \end{gather*}
    Wäre $[2]=[0]$ müsste diese Gleichung für alle Punkte von $E$
    bzw. alle $x\in\algK$ gelten, woraus $b_2=0=b_6$ folgen müsste.
    Dies impliziert nach den Formeln für die Diskriminante 
    \cite{silverman}[Proposition A.1.1],
    dass diese gleich Null ist und somit $E$ singulär
    \cite{silverman}[Proposition A.1.2 (a)].
    D.\,h. $E$ wäre entgegen der Voraussetzung keine elliptische
    Kurve.
    
    Den Fall ungerader $m\in\Z$ und $\Char(\algK)\neq2$ zeigt man mit
    der Konstruktion eines nicht trivialen Punktes $\O\neq P\in E$,
    für den $P+P=\O$ ist. Entsprechend ist dann
    $[2m+1](P)=[m]\circ[2](P)+P=P\neq \O$.
    
    Alle anderen Fälle ergeben sich wegen
    $[m\cdot n]=[m]\circ[n]$ aus dem Fall für $[2]$ und dem für ungerade
    $m\in\Z$.
  \end{proof}
\end{Lemma}

\begin{Korollar}
  Seien $E_1$, $E_2$, $E$ elliptische Kurven. Dann ist $\Hom(E_1,E_2)$
  ein torsionsfreier $\Z$-Modul, d.\,h. für jede Isogenie
  $\phi\in\Hom(E_1,E_2)$ ist $[m]\circ\phi\neq[0]$ für
  $m\in\Z\setminus\{0\}$.

  Außerdem ist $\End(E,E)$ nullteilerfrei und hat Charakteristik Null.
  Beachte, dass $\End(E,E)$ ist nicht immer kommutativ ist!
  \begin{proof}
    \cite[siehe][Proposition 4.2 (b)]{silverman}
    
    Angenommen es gäbe ein $\phi\in\Hom(E_1,E_2)$ mit $m$-Torsion für
    ein $m\in\Z\setminus\{0\}$, d.\,h. 
    \begin{gather*}
      [0]
      =\underbrace{\phi+\dotsc+\phi}_{m-\text{mal}}
      = [m]\circ\phi
    \end{gather*}
    ($[0]\colon E_1\to E_2$ hier die konstante Isogenie).
    Nachdem der Grad von Körpererweiterungen multiplikativ ist, ist es
    per Definition auch der von Morphismen:
    \begin{gather*}
      0 \eqqcolon \deg([0]) = \deg([m])\cdot\deg(\phi)
    \end{gather*}
    Nach \autoref{additionnichtkonstant} ist $[m]$ nicht konstant,
    also auch $\deg([m])>0$ für $m\neq0$. Daher muss $\deg(\phi)=0$
    und somit $\phi=[0]$ bereits konstant gewesen sein.

    Daraus folgt bereits, dass $\End(E)=\Hom(E,E)$ Charakteristik Null hat.
    Außerdem gilt wieder mit der Multiplikativität der Grade für zwei
    Abbildungen $\phi,\psi\in\End(E)$
    \begin{gather*}
      \phi\circ\psi=[0]
      \qquad\Longrightarrow\qquad
      0 = \deg(\phi\circ\psi) = \deg(\phi)\cdot\deg(\psi)
    \end{gather*}
    also ist entweder $\phi$ oder $\psi$ konstant.
    Damit ist $\End(E)$ nullteilerfrei.
  \end{proof}
\end{Korollar}

\begin{Definition}[$m$-Torsionsuntergruppe]
  Für einen elliptische Kurve $E$ und $m\in{\Z\setminus\{0\}}$ ist die
  $m$-Torsionsuntergruppe von $E$
  \begin{gather*}
    E[m] \coloneqq \left\{ P\in E\;\middle|\; [m](P)=\O\right\}
    = E[-m]
    \overset{\autoref{additionnichtkonstant}}{\neq}E
  \end{gather*}
  Die Torsionsuntergruppe von $E$ ist die Vereinigung
  \begin{gather*}
    E_\text{tors} \coloneqq \bigcup_{\mathclap{m=1}}^{\infty}E[m]
  \end{gather*}
\end{Definition}



% ---

\section{Eigenschaften von Isogenien als Gruppenhomomorphismen}
In diesem Kapitel werden wir sehen, dass Isogenien nicht nur
Morphismen von Kurven sondern sogar Gruppenhomomorphismen von
Punktegruppen sind.
Diese Eigenschaft liefert eine Verbindung von den 
rationalen Abbildungen zur Gruppenstruktur auf elliptischen
Kurven.

\subsection{Isogenien als Gruppenhomomorphismen}
\begin{Satz}\label{isogenienhoms}
  Isogenien sind Homomorphismen zwischen den Punktgruppen von
  elliptischen Kurven, d.\,h. für eine Isogenie $\phi\colon E_1\to
  E_2$ gilt
  \begin{gather*}
    \phi(P+Q) = \phi(P) + \phi(Q)
    \qquad \text{für alle } P,Q\in E_1
  \end{gather*}
  \begin{proof}
    Für die konstante Abbildung $\phi\equiv \O$ ist es klar.
    Ist $\phi$ nicht konstant, machen wir uns die Gruppenisomorphie
    zwischen der Grad-0-Untergruppe der Picard-Gruppe und der
    Punktegruppe einer elliptischen Kurve zunutze
    (s.\,\autoref{gruppenaequivalenz}).
    \autoref{gruppenaequivalenz}\,i) liefert die Gruppenisomorphismen 
    \begin{alignat*}{3}
      &\kappa_1\colon&
      E_1&\overset\sim\longto \Pic^0(E_1)
      &\qquad P &\longmapsto \overline{(P)-(\O)}\\
      &\kappa_2\colon&
      E_2&\overset\sim\longto \Pic^0(E_2)
      &P &\longmapsto \overline{(P)-(\O)}\\
    \end{alignat*}
    und \autoref{gruppenaequivalenz}\,ii) den zu $\phi$
    korrespondierenden Gruppenhomomorphismus auf den Grad-0-Untergruppen
    der Picard-Gruppen
    \begin{align*}
      \phi_*\colon \Pic^0(E_1) 
      &\longto \Pic^0(E_2)\\
      \overline{\sum_{\mathclap{P\in E_1}}n_P\cdot(P)}
      &\longmapsto 
        \overline{\sum_{\mathclap{P\in E_1}}n_P\cdot(\phi(P))}
        \qquad .
    \end{align*}
    Insgesamt erhalten wir das kommutative Diagramm
    \begin{center}
      \begin{tikzcd}
        E_1 
        \arrow[r, "{\kappa_1}"{below}, "\sim"{above}]
        \arrow[d,"\phi"{left}] 
        & \Pic^0(E_1) 
        \arrow[d, "\phi_*"{right}]
        \arrow[dl, phantom, "\circlearrowleft"{description}]
        \\
        E_2 
        \arrow[r, "{\kappa_2}"{below}, "\sim"{above}]
        & \Pic^0(E_2) 
      \end{tikzcd}
    \end{center}
    welches zeigt, dass $\phi=\kappa_2^{-1}\circ\phi_*\circ\kappa_1$
    ist, also ein Gruppenhomomorphismus.
  \end{proof}
\end{Satz}

% ---

\subsection[Kerne]{Kerne von Isogenien}
Aus \autoref{kategorienaequivalenz} wissen wir bereits, dass Isogenien
durch eine Kategorienantiäquivalenz als Körpererweiterungen aufgefasst
werden können. 
Jetzt werden wir mithilfe dieses Zusammenhangs eine schöne Darstellung
der Kerne von Isogenien kennen lernen und den Funktionenkörper einer
Isogenie genauer betrachten.
Diese Darstellung und die Antwort auf die Frage, wann eine Isogenie einer
Galoiserweiterung entspricht, werden in den Folgekapiteln sehr
nützlich sein, um weitere Eigenschaften von Isogenien als
Homomorphismen zu untersuchen. 

Wir beginnen damit, die Aussagen aus \autoref{sepgrad} zum
Verzweigungsindex zu erweitern.
\begin{Lemma}\label{sepgrad2}
  Sei $\phi\colon E_1\to E_2$ eine nicht-konstante Isogenie.
  \begin{enumerate}[label=\roman*)]
  \item Für alle $Q\in E_2$ gilt
    \begin{gather*}
      \#\phi^{-1}(Q) = \degs(\phi)
    \end{gather*}
    Insbesondere gilt für die Mächtigkeit des Kerns von $\phi$ als
    Gruppenhomomorphismus 
    $\#\ker\phi=\#\phi^{-1}(\O)=\degs(\phi)$.
  \item Für alle $Q\in E_1$ gilt
    \begin{gather*}
      e_{\phi}(Q) = \degi(\phi)
    \end{gather*}
  \end{enumerate}
  \begin{proof}~
    \begin{enumerate}[label=\roman*)]
    \item Für diesen Teil müssen wir nur die Aussage aus
      \autoref{sepgrad}\,ii) erweitern, die besagt, dass die Behauptung
      in \emph{i)} von oben für alle außer endlich viele Punkte in $E_2$
      gilt. Dazu zeigen wir, dass $\#\phi^{-1}(Q)$ für alle Punkte $Q\in
      E_2$ gleich ist. 
      Betrachte also $Q,Q'\in E_2$. Nachdem $\phi$ surjektiv ist
      (siehe \autoref{morphismensurj}), gibt es einen Punkt $R\in E_1$ mit
      \begin{gather*}
        \phi(R) = Q' - Q
      \end{gather*}
      Nun können wir nutzen, dass $\phi$  nach \autoref{isogenienhoms} als
      Isogenie ein Gruppenhomomorphismus zwischen den Punktegruppen von
      $E_1$ und $E_2$ ist. Wir erhalten, dass die Translationen
      \begin{align*}
        \phi^{-1}(Q) &\longto \phi^{-1}(Q')
        &\phi^{-1}(Q) &\longfrom \phi^{-1}(Q')\\
        P &\longmapsto P+R
        &P'-R &\longmapsfrom P'
      \end{align*}
      wohldefinierte, zueinander inverse Bijektionen zwischen diesen
      endlichen (siehe \autoref{sepgrad}\,\emph{i)}) Mengen sind.
      Denn für jedes $P\in\phi^{-1}(Q)$ ist
      $\phi(P+R)=\phi(P)+\phi(R)=Q'$, d.\,h. $(P+R)\in\phi^{-1}(Q')$,
      und umgekehrt.
      Damit braucht es nur einen Punkt $P\in E_2$ geben, für den Aussage
      \emph{i)} gilt. Das ist wegen \autoref{sepgrad}\,\emph{i)} erfüllt,
      wenn $E$ unendlich viele Punkte hat.
      Es hat aber jede Varietät $V/\algK$ positiver Dimension
      über einem algebraisch abgeschlossenen Körper dieselbe
      Kardinalität wie $\algK$, also mindestens abzählbar-unendlich
      \cite[siehe][Exercise I.4.8 oder Section I.6]{hartshorne}.
      
    \item Hier benötigen wir für unsere Rechnung die Multiplikativität
      des Verzweigungsindex aus \autoref{sepgrad}\,iii), um zu zeigen,
      dass auch alle Punkte in $\phi^{-1}(Q)$ den gleichen
      Verzweigungsindex bzgl. $\phi$ haben.
      Betrachte $P,P'\in E_1$ im Urbild $\phi^{-1}(Q)$. Das Bild ihrer
      Differenz $R\coloneqq P'-P$ ist folglich
      $\phi(P'-P)=\phi(P')-\phi(P)=\O$, weshalb die Verknüpfung
      $\phi\circ\tau_R = \phi$ mit der Translation um $R$ wieder $\phi$
      ist.
      Dann gilt für die Verzweigungsindizes aufgrund der
      Multiplikationsregel und mit $e_{\tau_R}\equiv1$ ($\tau_R$ ist
      Isomorphismus):
      \begin{gather*}
        e_\phi(P)
        = e_{\phi\circ\tau_R}(P)
        = e_{\tau_R}(P) \cdot e_\phi(\tau_R(P))
        = e_\phi(\tau_R(P))
        = e_\phi(P+R)
        = e_\phi(P')
      \end{gather*}
      Nachdem mit dieser Begründung jeder Punkt in $\phi^{-1}(Q)$
      denselben Verzweigungsindex hat, kann man folgendes Produkt
      umschreiben:
      \begin{align*}
        \degs(\phi) \cdot \degi(\phi)
        &= \deg(\phi)\\
        &\overset{\mathllap{\ref{sepgrad}\,i)}}{=}
          \sum_{\mathclap{P\in\phi^{-1}(Q)}} e_\phi(P)\\
        &\overset{\mathllap{\text{für alle Punkte gleich}}}{=}
          \left( \#\phi^{-1}(Q) \right) \cdot e_\phi(P)\\
        &\overset{\mathllap{\text{\emph{i)}}}}{=}
          \degs(\phi)\cdot e_\phi(P)
      \end{align*}
      Kürzen liefert die Behauptung \emph{ii)}.
    \end{enumerate}  
  \end{proof}
\end{Lemma}

Nun können wir den Kern einer Isogenie angenehm durch einen
Gruppenisomorphismus zu Körpererweiterungen darstellen.

\begin{Satz}\label{galois}
  Sei $\phi\colon E_1\to E_2$ wieder eine nicht-konstante Isogenie.
  \begin{enumerate}[label=\roman*)]
  \item Sei für einen Punkt $T\in E_1$ wieder 
    $\tau_T\colon P\mapsto P+T$ die Translationsabbildung und 
    $\tau_T^*$ die induzierte Abbildung auf $\algK(E_1)$.
    Dann ist
    \begin{align*}
      \Omega\colon
      \ker(\phi) 
      &\longto \Aut_{\phi^*(\algK(E_2))}\left( \algK(E_1) \right) \\
      T 
      &\longmapsto \left( \tau_T^* \colon f\mapsto f\circ \tau_T \right)
    \end{align*}
    ein Gruppenisomorphismus.
  \item Sei $\phi$ separabel 
    Dann ist $\phi$ unverzweigt,
    (d.\,h. $e_\phi(P)=1$ für alle $P\in E_1$),
    $\algK(E_1)$ ist eine Galoiserweiterung von $\phi^*(\algK(E_2))$
    und
    \begin{gather*}
      \#\ker(\phi) = \deg(\phi)
    \end{gather*}
  \end{enumerate}
  \begin{proof}~
    \begin{enumerate}[label=\roman*)]
    \item 
      \begin{description}
      \item[Wohldefiniertheit] $\tau_T$ ist bereits als Morphismus
        von Kurven ein Automorphismus von $E_1$ und damit ist
        $\tau_T^*$ ein Automorphismus von $\algK(E_1)$. Bleibt zu
        zeigen, dass $\phi^*(\algK(E_2))$ fixiert wird.
        Sei $T\in\ker(\phi)$, dann ist $\phi\circ\tau_T=\phi$.
        Für $f\in\algK(E_2)$ erhalten wir
        \begin{gather*}
          \tau_T^*(\phi^*(f)) = (\phi\circ\tau_T)^*(f) = \phi^*(f)
        \end{gather*}
        Also wird $\phi^*(\algK(E_2))$ fixiert.
      \item[Homomorphismus] Seien $S,T\in E_1$. Laut Definition ist
        klar, dass $\tau_S\circ\tau_T = \tau_{S+T}$. Also
        \begin{align*}
          \Omega(T+S)(f)
          &= \tau^*_{T+S}(f)\\
          &= f\circ\tau_{T+S} 
            = f\circ\tau_{S+T}
            = f\circ\tau_{S}\circ\tau_{T}\\
          &= \tau^*_T\circ\tau^*_S(f)
            = \Omega(T)\circ\Omega(S)(f)
        \end{align*}
      \item[Injektivität] Sei 
        $\tau_T^* = \Id_{\algK(E_1)} = \tau_\O^*
        \in\Aut_{\phi^*(\algK(E_2))}( \algK(E_1) )$.
        Das heißt für alle Funktionen $f\in\algK(E_1)$
        \begin{gather*}
          f 
          = f\circ\tau_\O
          = \tau_\O^*(f) 
          = \tau_T^*(f) 
          = f\circ\tau_T
        \end{gather*}
        Aus der Definition der rationalen Abbildungen $\tau_T$ und
        $\tau_\O$ \cite[siehe][Group Law Algorith III.2.3(c)]{silverman},
        die stark von den Koordinaten von $T$ und $\O$ abhängen,
        geht hervor, dass dann bereits die Koordinaten von
        $T$ und $\O$ übereinstimmen. Damit muss $T=\O$ sein.
      \item[Surjektivität]
        Für eine vereinfachte Schreibweise sei
        $\K_1\coloneqq\algK(E_1)$ und
        $\K_2\coloneqq{\phi^*(\algK(E_2))}$ der von $\phi$
        induzierte Unterkörper.
        Nach Algebra kann jeder $\K_2$-Automorphismus von $\K_1$
        erweitert werden zu einer $\K_2$-Einbettung von $\K_1$ in den
        algebraischen Abschluss $\overline{\K_2}$. Also gilt
        folgende Ungleichung 
        \begin{align*}
          \#\Aut_{\K_2}\left( \K_1 \right) 
          &\leq \#\Hom_{\K_2}\left( \K_1, \overline{\K_2} \right)\\
          &\eqqcolon
            \degs\left(\K_1/\K_2\right)\\
          &\eqqcolon \degs(\phi)
        \end{align*}
        Insgesamt gilt mit der Aussage $\degs(\phi)=\#\phi^{-1}(\O)$
        aus \autoref{sepgrad2}\,\emph{i)}
        \begin{gather*}
          \#\Aut_{\K_2}\left( \K_1 \right) 
          \leq \degs(\phi)
          \overset{\autoref{sepgrad2}\emph{i)}}{=} \#\phi^{-1}(\O)
          =\#\ker(\phi)
        \end{gather*}
        Nachdem es sich um eine endliche Erweiterung handelt, sind
        alle Teile der Ungleichung endlich und wir erhalten die
        Surjektivität von $\Omega$ mit der Injektivität von $\Omega$
        und dem Schubladenprinzip.
      \end{description}
    \item Sei $\phi$ separabel. Das heißt per Definition
      $\deg(\phi)=\degs(\phi)$.
      \begin{description}
      \item[Kernmächtigkeit]
        Mit \autoref{sepgrad2}\,\emph{i)} gilt
        $
        \#\ker(\phi)
        \coloneqq \#\phi^{-1}(\O)
        \overset{\mathclap{\autoref{sepgrad2}\emph{i)}}}{=} \degs(\phi)
        \overset{{\text{sep.}}}{=} \deg(\phi)
        $.
      \item[Unverzweigtheit] 
        Entsprechend ist
        $\degi(\phi)=\frac{\deg(\phi)}{\degs(\phi)}=1$
        und mit \autoref{sepgrad2}\,\emph{ii)} gilt
        $e_\phi(P)=1$ für alle $P\in E_1$, d.\,h. $\phi$
        ist unverzweigt.
      \item[Galoiserweiterung]
        Mit \autoref{sepgrad2}\,\emph{i)} und \autoref{galois}\emph{i)} gilt
        \begin{align*}
          \#\Aut_{\phi^*(\algK(E_2))}\left( \algK(E_1) \right) 
          \overset{\mathclap{\autoref{galois}\emph{i)}}}{=}
          \#\ker(\phi)
          \overset{{\text{s.o.}}}{=} \deg(\phi)
          \coloneqq \left[\algK(C_1):\phi^*(\algK(C_2))\right]
        \end{align*}
        was äquivalent zu
        $\algK(C_1)/\phi^*(\algK(C_2))$ galoissch ist.
      \end{description}
    \end{enumerate}
  \end{proof}
\end{Satz}

\subsection[Homomorphiesatz]{Homomorphiesatz für Isogenien}
Mit dem Zusammenhang von Kernen von Isogenien mit
Automorphismengruppen von Körpererweiterungen sowie der Aussage zu
galoisschen Isogenien können wir im Folgenden die Eigenschaften von
Isogenien als Homomorphismen weiter untersuchen.
Dazu vorerst eine Erinnerung an den Homomorphiesatz.

\begin{Bemerkung}[Homomorphiesatz]
  Der Homomorphiesatz besagt, dass es für
  einen Gruppenhomomorphismus $\alpha\colon G\to H$ einen eindeutigen
  Isomorphismus 
  $\bar\alpha\colon G/\ker\varphi\cong \im\varphi\subset H$ gibt, und
  es kommutiert das Diagramm
  \begin{center}
    \begin{tikzcd}
      G\arrow[r,"\alpha"]\arrow[d,"\text{proj}"{left}]&H\\
      G/\ker(\alpha)\arrow[ur, dashed, "\exists!\; \bar\alpha"{swap}]
    \end{tikzcd}
  \end{center}
  Also gibt es für Homomorphismen $\alpha\colon E_1\to E_2$,
  $\beta\colon E_1\to E_3$ ($E_i$ hier allg. Gruppen) einen eindeutigen
  Homomorphismus $\bar\lambda\colon \im(\alpha)\to\im(\beta)$, 
  sodass das folgende Diagramm kommutiert:
  \begin{center}
    \begin{tikzcd}[row sep=tiny]
      & E_1/\ker(\alpha)
      \arrow[r, "\bar\alpha"{below}, "\sim"{above}]
      \arrow[dd, "\text{proj}", twoheadrightarrow]
      & \im(\alpha)\subset E_2 
      \arrow[dd, dashed, twoheadrightarrow, "\exists!\;\lambda"]
      \\
      E_1
      \arrow[drr, "\beta"{near start, swap}, bend right=40]
      \arrow[urr, "\alpha"{near start}, bend left=40]
      \arrow[ur, twoheadrightarrow]
      \arrow[dr, twoheadrightarrow]
      \\
      ~ & E_1/\ker(\beta)
      \arrow[r, "\bar\beta"{below}, "\sim"{above}]
      & \im(\beta)\subset E_3
    \end{tikzcd}
    \hfill
    in kurz
    \hfill
    \begin{tikzcd}
      E_1
      \arrow[r, "\alpha"]
      \arrow[dr, "\beta"{swap}]
      & \im(\alpha)
      \arrow[d, "\exists!\;\lambda"]
      \arrow[r, hookrightarrow]
      & E_2\\
      & E_3
    \end{tikzcd}
  \end{center}
  Ist $\alpha$ surjektiv erhält man einen Homomorphismus
  $\lambda\colon E_2\to E_3$.
\end{Bemerkung}

Sind die $E_i$ Punktgruppen zu elliptischen Kurven 
und $\alpha$, $\beta$ nicht konstante Isogenien (d.\,h. insb. surjektiv),
stellt sich die Frage, in welchen Fällen der resultierende
Homomorphismus $\lambda\colon E_2\to E_3$ wieder eine Isogenie ist.

\begin{Satz}\label{homomorphiesatz}
  Sei $\phi\colon E_1\to E_2$ eine separable Isogenie.
  Jede weitere Isogenie $\psi\colon E_1\to E_3$ mit
  \begin{gather*}
    \SwapAboveDisplaySkip
    \ker\phi \subset \ker\psi
  \end{gather*}
  faktorisiert eindeutig über $\psi$, d.\,h. es gibt eine eindeutige
  Isogenie $\lambda\colon E_2\to E_3$ mit
  $\psi = \lambda\circ\phi$
  bzw. das folgende Diagramm kommutiert
  \begin{center}
    \begin{tikzcd}
      E_1
      \arrow[dr,"\psi"{swap}]
      \arrow[r,"\phi"]
      & E_2
      \arrow[d, dashed, "\exists!\,\lambda"]\\
      & E_3
    \end{tikzcd}
  \end{center}
  \begin{proof}
    \autoref{galois}\,ii) liefert uns, dass
    $\algK(E_1)/\phi^*(\algK(E_2))$ eine Galoiserweiterung ist mit
    Galoisgruppe $\Aut_{\phi^*(\algK(E_2))}(\algK(E_1))$.
    Der Isomorphismus aus \autoref{galois}\,i) zeigt, dass sie aus
    Elementen der Form $\tau_T^*$ mit $T\in\ker(\phi)$ besteht.
    Da nach Voraussetzung $\ker(\phi)\subset\ker(\psi)$ gilt,
    ist auch
    \begin{align*}
      \Aut_{\phi^*(\algK(E_2))}(\algK(E_1))
      &= \left\{ \tau_T^* \;\middle|\; T\in\ker(\phi) \right\} \\
      &\subset
        \left\{ \tau_T^* \;\middle|\; T\in\ker(\psi) \right\}
        = \Aut_{\psi^*(\algK(E_3))}(\algK(E_1))
    \end{align*}
    D.\,h. die Galoisgruppe fixiert den Unterkörper
    $\psi^*(\algK(E_3))$, weshalb er nach Galoistheorie ein
    Unterkörper von $\phi^*(\algK(E_2))$ sein muss.
    Wir erhalten folgende Kette von Körpererweiterungen
    \begin{gather*}
      \psi^*(\algK(E_3)) 
      \subset \phi^*(\algK(E_2)) 
      \subset \algK(E_1)
    \end{gather*}
    Da $\phi^*$ injektiv und 
    $\phi^*\colon\algK(E_2)\to\phi^*(\algK(E_2))$ ein
    Körperisomorphismus ist, erhalten wir eine Körpererweiterung
    \begin{gather*}
      \iota\coloneqq 
      (\phi^*)^{-1} \circ \psi^*
      \colon \algK(E_3) \longto \algK(E_2)
    \end{gather*}
    Die Kategorienantiäquivalenz bzw. \autoref{erweiterungzuratabb}
    liefert jetzt einen Morphismus $\lambda\colon E_3\to E_2$ glatter
    Kurven mit
    \begin{gather*}
      \lambda^* = \iota = (\phi^*)^{-1} \circ \psi^*
      \qquad\text{bzw.}\qquad
      \phi^*\circ\lambda^* = \psi^*
    \end{gather*}
    also wie gesucht $\lambda\circ\phi = \psi$.
    Es bleibt zu zeigen, dass $\lambda$ eine Isogenie ist:
    \begin{gather*}
      \lambda(\O) 
      \overset{\text{$\phi$ Isog.}}{=} \lambda(\phi(\O))
      = \psi(\O) 
      \overset{\text{$\psi$ Isog.}}{=} \O
    \end{gather*}
  \end{proof}
\end{Satz}


\subsection[Quotientenbildung]{Quotientenbildung auf elliptischen Kurven}
Zum Schluss werden wir noch sehen, dass bestimmte Quotienten der
Punktegruppe einer elliptischen Kurve wieder die Struktur einer
elliptischen Kurve haben, wobei der zugehörige
Projektionshomomorphismus eine (sogar separable) Isogenie ist.

\begin{Satz}
  Sei $E$ eine elliptische Kurve und $\F\subset E$ eine endliche
  Untergruppe.
  Dann gibt es eine eindeutige elliptische Kurve $E'$ und eine
  separable, unverzweigte Isogenie $\phi\colon E\to E'$, so dass
  \begin{gather*}
    \ker\phi = \F \;.
  \end{gather*}
  $E'$ wird auch $E/\F$ geschrieben.

  Ist $E$ definiert über $K$ und $\F$ invariant bzgl. der Galoisgruppe
  von $\algK/\K$, so können $E'$ und $\phi$ definiert über $K$ gewählt
  werden.
  \begin{proof}
    Zuerst benötigen wir einen geeigneten Unterkörper von $\algK(E)$
    als Funktionenkörper vom gesuchten $E'$.
    Wie in \autoref{galois}\,\emph{i)} erhalten wir zu jedem Punkt
    $T\in \F$ einen Automorphismus $\tau_T^*$ von $\algK(E)$.
    Bezeichne die Untergruppe dieser Automorphismen mit 
    \begin{gather*}
      \Phi\coloneqq\left\{ \tau_T^* \;\middle|\; T\in \F \right\}
      \cong F
    \end{gather*}
    und betrachte den Fixkörper $\algK(E)^\Phi$.    
    $\algK(E)/\algK(E)^\Phi$ ist eine Galoiserweiterung
    (insb. separabel) mit Galoisgruppe $\Phi$ und wir erhalten dank
    der Kategorienantiäquivalenz aus \autoref{kategorienaequivalenz}
    eine eindeutige glatte Kurve $C/\algK$ zu $\algK(E)^\Phi$ und
    einen endlichen, separablen Morphismus $\phi\colon E\to C$ zur
    Erweiterung $\algK(E)^\Phi\to\algK(E)$ mit
    $\phi^*(\algK(C))=\algK(E)^\Phi$.
    
    Es bleibt zu zeigen, dass $\phi$ unverzweigt ist, und, dass $C$
    eine elliptische Kurve ist, d.\,h. Geschlecht $g_C=1$ hat.
    Letzteres können wir mithilfe der Hurwitz'schen Formel
    \autoref{hurwitz} zeigen.

    \begin{description}
    \item[$\phi$ unverzweigt]
      Es gilt wegen $e_\phi(P)\geq1$ für alle $Q\in E$ die Ungleichung
      \begin{gather*}
        \#\phi^{-1}(Q)
        \leq
        \sum_{\mathclap{P\in\phi^{-1}(Q)}} e_\phi(P)
        \overset{\text{\autoref{sepgrad}\,\emph{i)}}}{=}
        \deg(\phi)
        \overset{\text{Galois}}{=}
        \#\Phi
        =\# \F
      \end{gather*}
      und Gleichheit gilt genau dann, wenn $e_\phi(P)=1$ für alle
      $P\in C$, also wenn $\phi$ unverzweigt ist.
      Es ist also zu zeigen, dass $\#\phi^{-1}(Q)\geq\#\Phi=\# \F$ für alle 
      $Q\in E$.

      Sei $T\in\F$, dann ist $\tau_T^*\in\Phi$ die Identität auf
      $\phi^*(\algK(C))=\algK(E)^\Phi$. Daraus folgt
      \begin{gather*}
        f(\phi(P+T))
        = f(\phi\circ\tau_T(P)) 
        = \left( \tau_T^*(\phi^*(f)) \right)(P)
        % = (\tau_T^*(\underset{\mathrlap{\in\phi^*(\algK(C))}}{\phi^*(f)}))(P)
        = \phi^*(f)(P)
        = f(\phi(P))
      \end{gather*}
      für alle $P\in E$ und $f\in\algK(C)$.
      Also $\phi(P) = \phi(P+T)$ für alle $P\in E$.
      Für ein $Q\in C$ und ein beliebiges $P\in\phi^{-1}(Q)$ gilt
      deshalb
      \begin{gather*}
        \phi^{-1}(Q) 
        \supset \left\{ P+T \;\middle|\; T\in\Phi \right\}        
      \end{gather*}
      Nachdem $P+T\neq P+T'$ für $T\neq T'\in\F$, gilt für die
      Kardinalitäten
      \begin{gather*}
        \#\phi^{-1}(Q) 
        \geq \#\left\{ P+T \;\middle|\; T\in F \right\}
        = \#\F
      \end{gather*}
      Damit ist $\deg(\phi)=\phi^{-1}(Q)$ für alle $Q\in E$ und
      $\phi$ ist unverzweigt.
      
    \item[$C$ elliptische Kurve]
      Nachdem $\phi\colon E\to C$ ein separabler Morphismus glatter
      Kurven ist, kann die Hurwitz'sche Formel \autoref{hurwitz}
      angewendet werden ($g_E=1$, da $E$ elliptische Kurve, $g_C$ ist das
      gesuchte Geschlecht von $C$):
      \begin{align*}
        0
        = 2\underset{=1}{g_E}-2
        &\overset{\text{Hurwitz}}{\geq}
          \underbrace{\deg(\phi)}_{>0} (2g_C-2) 
          + \sum_{\mathclap{P\in C}}\underbrace{(e_P(\phi)-1)}_{=0}
          \geq 0\\
      \end{align*}
      Also ist das Geschlecht von $C$ gleich $g_C=1$ und $C$ mit
      neutralem Element $\O\in E$ ist elliptische Kurve.
    \end{description}
  \end{proof}

\end{Satz}


\nocite{*}
\printbibliography
\end{document}
